\documentclass[12pt]{book}              % Book class in 11 points
\parindent0pt  \parskip10pt             % make block paragraphs
\usepackage{hyperref}
%\raggedright                            % do not right justify

\title{\bf Family Portrait}    % Supply information
\author{J. A. Morris}              %   for the title page.
\date{New York, 1962}                           %   Use current date. 

% Note that book class by default is formatted to be printed back-to-back.
\begin{document}                        % End of preamble, start of text.
\frontmatter                            % only in book class (roman page #s)
\maketitle                              % Print title page.
\tableofcontents                        % Print table of contents
\mainmatter                             % only in book class (arabic page #s)
\part{A Part Heading}                   % Print a "part" heading
\chapter*{Forward}                % Print a "chapter" heading
%Most of this example applies to \texttt{article} and \texttt{book} classes
%as well as to \texttt{report} class. In \texttt{article} class, however,
%the default position for the title information is at the top of
%the first text page rather than on a separate page. Also, it is
%not usual to request a table of contents with \texttt{article} class.
% 
%\section{A Subheading}                  % Print a "section" heading
%The following sectioning commands are available:
%\begin{quote}                           % The following text will be
% part \\                                %    set off and indented.
% chapter \\                             % \\ forces a new line
% section \\ 
% subsection \\ 
% subsubsection \\ 
% paragraph \\ 
% subparagraph 
%\end{quote}                             % End of indented text
%But note that---unlike the \texttt{book} and \texttt{report} classes---the
%\texttt{article} class does not have a ``chapter" command.

I have always had the urge to write something of longtime interest to others. During a long business career I did a lot of writing -- studies, reports, suggestions -- of the current interest only. They were communications that seemed very important at the moment. But I daresay very few could be found at this writing (January, 1962). If available, they would have little meaning now. Perhaps a suggestion on business policy is still timely and hauled out of the back of an office file for reference. Of, maybe, a letter to family or friends contained a helpful comment that warranted storing in that packet of memories in the attic trunk. However, there are few communications in the average person's life that stand the test of time.

This was emphasized very strongly in my attempt to reconstruct the family genealogy. The record goes back for more than 300 years. But it is a record of births and deaths with a bare sprinkling of facts on which to evaluate the character, life and works of the individuals. Perhaps the urge to set down some words that would be of enduring value as a family portrait sketch for the younger offshoots is due to the fact that I, like others of my generation, will also be a mere statistic before many years. This urge to preserve a background of the family is much stronger than my lifetime reluctance to look back. 

[Illegible]

So, here is the beginning of a story that might, I hope, give life to some of the statistics on the recent generations of my family. Because it must necessarily revolve around the one I know most about, it is written in the first person. 

\chapter{The Story Begins}

For me, the story begins with a statistic -- April 6, 1893 -- in a two-story and basement frame house on Buffalo Avenue, Brooklyn. Around the corner at 844 Herkimer Street in the backyard garden of a more pretentious brick house, my father's young cousin, Caroline Halliday, was playing and keeping a watchful eye on the upper rear windows of the Buffalo Avenue House. She was watching for a signal that her mother, Grandma Morris' sister, was needed. The signal, a sheet hung out a window, came and aunt Addi rushed to the Buffalo Avenue house and helped with my birth. Eighteen months earlier my parents had welcomed their first child -- Chester DeVere -- my big brother to the world. Four years later in 1897 my sister, Voda was born in the Buffalo Avenue house.

My father, Isaac J. Morris, was 25 years of age when I was born and my mother, Birdella LaBonte Morris, 23. Whether they wanted me so soon after the first child is a moot question. But I came along anyway and Mom told me in later years that I was a cry baby, a thumb sucker and an apron-string hanger-on until I was old enough to go out and rough-it-up with my playmates. I do know that I was strongly attached to her. 

I can still recall the misery experienced when she left me for the first time to visit her parents in Albany. We lived in Morris Park then and the memory of my walk with her to the railroad station is vivid. I was crying and she would pause to dry my eyes and console me. To this day, the sound of a train whistle in the night brings back the acute feeling of loneliness I felt that first night she was away. 

I have no recollection of my babyhood in the Buffalo Avenue house. All of my memories of the neighborhood stem from childhood visits to Herkimer Street with my parents after we had moved to Morris Park. Aunt Addie was a widow with three children -- Isaac, Carrie and Walter. Grandma Morris (Ann), also a widow, lived next door with her brother, David Swayze. At that time the neighborhood was upper middle class. To us it seemed like millionaire's row for we were quite poor. The street was cobble-stoned and I can still hear the click-click of horse shoes and the lurching screech of iron-rimmed wagon wheels on the stones. 

The gardens in the rear yards stretched back for maybe 150 feet and we enjoyed playing in them. There were a couple of peach trees with low-hung branches that tempted us to snatch at the big, luscious fruit when Aunt Addie could not see us. 

The Haliday house was a lively one. The two boys were always playing pranks on their mother and sister, and on Chester and me when we visited them. Cousin Ike was the older and more serious of the two and highly opinionated -- a fact that made him rather forbidding in later life. Walter and Carrie were really fun-loving and laughter was the order of the day -- even more so when Arther Blinn, Pop's nephew, visited the Halidays. I think Arthur's mother, Caroline Morris Blinn, died during his birth. It was a home we loved to visit. But it was not without its tragedy. 

Aunt Addie had the problem of raising three children. To help bolster a limited income, she took in boarders. One of them, a handsome real estate broker, courted and married her in 1901. A short time later he disappeared and was not heard of again. Although she never said so, we believe she expected he would return and for that reason has remained in the Herkimer Street house. At this writing, she is well into her eighties and still there in a much run-down neighborhood -- a sad contrast to its stately refinement of her childhood. 

Despite little formal education, Cousin Ike was successful in business. He was associated with an estate and at one time as part of his responsibilities managed the \href{http://en.wikipedia.org/wiki/Hotel_Theresa}{Hotel Theresa} in Harlem. That section of Harlem, now entirely negro, was a very fine neighborhood. It was really a big treat to dine with Cousin Ike and his wife Birdie. Mother would polish us up, put on her very best clothes and admonish us on our table manners. Cousin Ike later became night manager of the New York Stock Exchange Clearing House, a job he kept until his retirement. 

Grandma Morris, Uncle Dave and Aunt Addie have long since passed away. More recently, Cousins Arthur, Ike and Walter have died. Walter's son, Walter J., is living in Rockville Centre. 

As mentioned earlier, I have no recollection of my first home in Brooklyn. We moved to Morris Park (\href{http://en.wikipedia.org/wiki/Richmond_Hill,_Queens}{Richmond Hill}, south of the railroad) when I was about four years old. Pop was a machinist in the Long Island Railroad shops in that village. The new home was a far-from-pretentious flat in a three-story six-family house on the northwest corner of Briggs Ave (now 117th Street) and Chichester Avenue (now 95th Avenue). There was only one house directly opposite on the avenue. In the rear of our flat was a fenced-in yard and beyond that open fields almost to Atlantic Avenue. This is where we played ball when we were older. 

Morris Park was really a country village in those days. Five blocks south, Liberty Avenue was ``the end of the world.'' From there we walked through the woods (now Glen Morris) to the old water holes in the swamps of the Aqueduct. And we did not need bathing trunks for our swimming, the high grasses and cat-o-nine tails\footnote{a.k.a \href{http://en.wikipedia.org/wiki/Typha_latifolia}{\it Typha latifolia}, a wetland weed} hid us from view even if there was any one within hailing distance. 






\end{document}