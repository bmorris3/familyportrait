\documentclass[12pt]{book}              % Book class in 11 points
\parindent0pt  \parskip10pt             % make block paragraphs
\usepackage{hyperref}
\usepackage{wasysym} % Needed for cents symbol
\usepackage{float}
%\usepackage{footnote}
%\makesavenoteenv{tabular}
%\makesavenoteenv{table}
%\raggedright                            % do not right justify

%\newcommand{\supth}{$^{\mathrm{th}}$}

\title{\textsc{Family Portrait}\\ The Memoirs of James Alfred Morris}    % Supply information
\author{J. A. Morris}              %   for the title page.
\date{New York, 1962}                           %   Use current date. 

% Note that book class by default is formatted to be printed back-to-back.
\begin{document}                        % End of preamble, start of text.
\frontmatter                            % only in book class (roman page #s)

\section*{Note from the editor}
This is an lightly edited reproduction of the original typed manuscripts of the memoir of J.~A.~Morris, with hand-written annotations and corrections. It was transcribed by Brett M.~Morris, great-grandson of the author, in 2014. 

The annotations in the margins have faded in places and are illegible. When large bits of text are missing that cause breaks in the narrative, they will be marked with the word: [Illegible]. %Missing pages will be marked similarly.

In an attempt to remain true to the manuscripts, any descriptions or definitions added by the editor to clarify some of the more dated language are written in footnotes. 



\maketitle                              % Print title page.
\tableofcontents                        % Print table of contents
\mainmatter                             % only in book class (arabic page #s)
%\part{A Part Heading}                   % Print a "part" heading
\chapter*{Forward}                % Print a "chapter" heading
%Most of this example applies to \texttt{article} and \texttt{book} classes
%as well as to \texttt{report} class. In \texttt{article} class, however,
%the default position for the title information is at the top of
%the first text page rather than on a separate page. Also, it is
%not usual to request a table of contents with \texttt{article} class.
% 
%\section{A Subheading}                  % Print a "section" heading
%The following sectioning commands are available:
%\begin{quote}                           % The following text will be
% part \\                                %    set off and indented.
% chapter \\                             % \\ forces a new line
% section \\ 
% subsection \\ 
% subsubsection \\ 
% paragraph \\ 
% subparagraph 
%\end{quote}                             % End of indented text
%But note that---unlike the \texttt{book} and \texttt{report} classes---the
%\texttt{article} class does not have a ``chapter" command.

I have always had the urge to write something of longtime interest to others. During a long business career I did a lot of writing -- studies, reports, suggestions -- of the current interest only. They were communications that seemed very important at the moment. But I daresay very few could be found at this writing (January, 1962). If available, they would have little meaning now. Perhaps a suggestion on business policy is still timely and hauled out of the back of an office file for reference. Of, maybe, a letter to family or friends contained a helpful comment that warranted storing in that packet of memories in the attic trunk. However, there are few communications in the average person's life that stand the test of time.

This was emphasized very strongly in my attempt to reconstruct the family genealogy. The record goes back for more than 300 years. But it is a record of births and deaths with a bare sprinkling of facts on which to evaluate the character, life and works of the individuals. Perhaps the urge to set down some words that would be of enduring value as a family portrait sketch for the younger offshoots is due to the fact that I, like others of my generation, will also be a mere statistic before many years. This urge to preserve a background of the family is much stronger than my lifetime reluctance to look back. 

[Illegible]

So, here is the beginning of a story that might, I hope, give life to some of the statistics on the recent generations of my family. Because it must necessarily revolve around the one I know most about, it is written in the first person. 

\chapter{The Story Begins}

For me, the story begins with a statistic -- April 6, 1893 -- in a two-story and basement frame house on Buffalo Avenue, Brooklyn. Around the corner at 844 Herkimer Street in the backyard garden of a more pretentious brick house, my father's young cousin, Caroline Halliday, was playing and keeping a watchful eye on the upper rear windows of the Buffalo Avenue House. She was watching for a signal that her mother, Grandma Morris' sister, was needed. The signal, a sheet hung out a window, came and Aunt Addie rushed to the Buffalo Avenue house and helped with my birth. Eighteen months earlier my parents had welcomed their first child -- Chester DeVere -- my big brother to the world. Four years later in 1897 my sister, Vada was born in the Buffalo Avenue house.

My father, Isaac J. Morris, was 25 years of age when I was born and my mother, Birdella LaBonte Morris, 23. Whether they wanted me so soon after the first child is a moot question. But I came along anyway and Mom told me in later years that I was a cry baby, a thumb sucker and an apron-string hanger-on until I was old enough to go out and rough-it-up with my playmates. I do know that I was strongly attached to her. 

I can still recall the misery experienced when she left me for the first time to visit her parents in Albany. We lived in Morris Park then and the memory of my walk with her to the railroad station is vivid. I was crying and she would pause to dry my eyes and console me. To this day, the sound of a train whistle in the night brings back the acute feeling of loneliness I felt that first night she was away. 

I have no recollection of my babyhood in the Buffalo Avenue house. All of my memories of the neighborhood stem from childhood visits to Herkimer Street with my parents after we had moved to Morris Park. Aunt Addie was a widow with three children -- Isaac, Carrie and Walter. Grandma Morris (Ann), also a widow, lived next door with her brother, David Swayze. At that time the neighborhood was upper middle class. To us it seemed like millionaire's row for we were quite poor. The street was cobble-stoned and I can still hear the click-click of horse shoes and the lurching screech of iron-rimmed wagon wheels on the stones. 

The gardens in the rear yards stretched back for maybe 150 feet and we enjoyed playing in them. There were a couple of peach trees with low-hung branches that tempted us to snatch at the big, luscious fruit when Aunt Addie could not see us. 

The Haliday house was a lively one. The two boys were always playing pranks on their mother and sister, and on Chester and me when we visited them. Cousin Ike was the older and more serious of the two and highly opinionated -- a fact that made him rather forbidding in later life. Walter and Carrie were really fun-loving and laughter was the order of the day -- even more so when Arther Blinn, Pop's nephew, visited the Halidays. I think Arthur's mother, Caroline Morris Blinn, died during his birth. It was a home we loved to visit. But it was not without its tragedy. 

Aunt Addie had the problem of raising three children. To help bolster a limited income, she took in boarders. One of them, a handsome real estate broker, courted and married her in 1901. A short time later he disappeared and was not heard of again. Although she never said so, we believe she expected he would return and for that reason has remained in the Herkimer Street house. At this writing, she is well into her eighties and still there in a much run-down neighborhood -- a sad contrast to its stately refinement of her childhood. 

Despite little formal education, Cousin Ike was successful in business. He was associated with an estate and at one time as part of his responsibilities managed the \href{http://en.wikipedia.org/wiki/Hotel_Theresa}{Hotel Theresa} in Harlem. That section of Harlem, now entirely negro, was a very fine neighborhood. It was really a big treat to dine with Cousin Ike and his wife Birdie. Mother would polish us up, put on her very best clothes and admonish us on our table manners. Cousin Ike later became night manager of the New York Stock Exchange Clearing House, a job he kept until his retirement. 

Grandma Morris, Uncle Dave and Aunt Addie have long since passed away. More recently, Cousins Arthur, Ike and Walter have died. Walter's son, Walter J., is living in Rockville Centre. 

As mentioned earlier, I have no recollection of my first home in Brooklyn. We moved to Morris Park (\href{http://en.wikipedia.org/wiki/Richmond_Hill,_Queens}{Richmond Hill}, south of the railroad) when I was about four years old. Pop was a machinist in the Long Island Railroad shops in that village. The new home was a far-from-pretentious flat in a three-story six-family house on the northwest corner of Briggs Ave (now 117\textsuperscript{th} Street) and Chichester Avenue (now 95\textsuperscript{th} Avenue). There was only one house directly opposite on the avenue. In the rear of our flat was a fenced-in yard and beyond that open fields almost to Atlantic Avenue. This is where we played ball when we were older. 

Morris Park was really a country village in those days. Five blocks south, Liberty Avenue was ``the end of the world.'' From there we walked through the woods (now Glen Morris) to the old water holes in the swamps of the Aqueduct. And we did not need bathing trunks for our swimming, the high grasses and cat-o-nine tails\footnote{a.k.a \href{http://en.wikipedia.org/wiki/Typha_latifolia}{\it Typha latifolia}, a wetland weed} hid us from view even if there was any one within hailing distance. 

Two events are the earliest recollection of my childhood. Both occurred at the first home in Morris Park. I was just a little toddler playing in the yard while Mom hung the wash. An Irish woman leaning out an upper window shouted that my pants were falling open (actual words censored). If a baby can be embarrassed, I was and so was Mom. It was peculiar that an incident like that can be remembered while other more important ones cannot. The other event was at a Thanksgiving or Christmas party in our flat. The Halidays and Grandma Morris were there and my little baby sister Vada was the center of attention. In some way, one of the glass dessert dishes was cracked and I swallowed a small chip of glass. Pandemonium broke loose with much shouting and wringing of hands probably brought about by my wailing. Finally some one thought of calling a doctor. He filled me full of crackers or bread, topped off with a large dose of castor oil. Evidently they found the glass. 

Pop was an apprentice machinist in the railroad shops at Albany or Rensselaer when he courted Mom. [Illegible] I do know that he worked hard and had long hours. As I recall, the deep-throated, long-carrying shop whistle awakened the entire village at seven each morning, blew at 12 noon for a half hour lunch period and at 6 p.m. for closing. That was a 10.5 hour day for the shop personnel. His pay check for a six day week was \$12 or \$15. Compared with present day hours and pay that certainly was unbelievable. 

It was a monotonous treadmill of work with little time left for anything else and with no future. Pop was handicapped by lack of formal education. He had not completed grammar school but was an excellent mechanic and of an inventive turn of mind. He sought to lift himself through inventions and by going into business for himself. He was not successful in either direction. 

At one time he felt there was a market for a convenient hand washing compound. From his daily experience he knew the difficulty of washing grease and grime from his hands at the end of a day's work. His idea was a compressed washing powder about the diameter of a quarter and a quarter inch thick, packed in rolls like the old candy \href{http://en.wikipedia.org/wiki/Necco_Wafers}{Necco Wavers}. Held in the palms of the hands under the water tap it would dissolve and with a little rubbing remove the grease. He advertised for shop workers with little success. [Illegible] Later the big soap companies introduced hand washing compounds in paste form which met with the success that Pop had so much sought. 

Another of his many inventions was a collapsible wooden crate for packing onions and other vegetables -- that could be returned to the farmer or produce dealer and reused. He patented the idea, incorporated a company under the name {\it American Crate Company} and sold stock to his friends. Then the promoter sold Pop on teh idea that he could sell the crate to the onion growers of Texas and departed with the remaining cash. Years later he sent shares of stock in an oil exploration company to Pop and to the stockholders of the Crate Company. They turned out to be as worthless as the crate company stock and ended another dream for Pop. 

We can return to our early days in Morris Park with Pop's venture in the retail grocery business. Three doors down the block from our flat was a little frame house, the front room of which had been converted into a store. Pop left the railroad, moved the family into the little house and opened a grocery store advertised as ``The Little Store with the Little Prices.'' One thing I recall about the house was the grape arbor forming a leafy corridor to a two-holer outhouse. Perhaps the corridor was not too long, but one time I didn't quite make its entire length much to the consternation of Mom and my own embarrassment. By way of extenuation it may be said that I was only four or five years old. 

A very short time later, Pop arranged to have a new store built on the lot immediately adjoining the little store. The structure was built on the entire width of the lot. It had a built-in driveway to a stable in the rear of the house. The store and rear storeroom took up the remainder of the first level. From the storeroom a staircase led to a very nice apartment above. Pop continued in business here for a few years until his leniency on credit brought the business toppling down.

An old oder book (I don't know if it was for delivery from the little store or from the new one) discloses some rather startling prices when compared with those today. Here are a few examples: 

\begin{table}[H]
\centering
\begin{tabular}{l r}
3.5 pounds of sugar & 17\cent \\
1 pound of coffee & 36\cent \\
1 loaf bread & 5\cent \\
1 can condensed milk & 10\cent \\
1 quart milk & 5\cent \\
1 gallon Kerosene oil & 13\cent \\
1 lamp wick & 2\cent \\
1 quart white onions & 12\cent \\
1 peck potatoes & 18\cent \\
1 bushel coal & 17\cent \\
\end{tabular}
\end{table}
%\footnotetext{1 peck of potatoes= 15 pounds}
%\footnotetext{1 bushel of coal = 80 pounds}

\chapter{Memory begins to crystallize}

It is from this place that my memory begins to crystallize. Names of playmates, the rough and tumble fights of childhood, the short cut across the fields to the primary school on Elm Street, and the many other things that made their impressions on a developing mind. I would like to mention a few incidents not only for the nostalgic interest they hold for me but as a sort of factual background for appraisal of the effect on our character. 

First, let me say that my sister Vada was born in this house in 1897. I don't remember very much about her entrance, only the few times Mom asked me to push the baby carriage. I do remember the admiration she drew from the relatives when they visited Morris Park. She was a beautiful baby and grew into a beautiful girl and woman.

Because Pop's name was Issac, two of the neighborhood boys delighted in calling me ``Ikey.'' One of the boys, George Washington, was colored and the other, Tommy Givins, Irish. They were a little older and I was dreadfully afraid of them for a long time. Finally the taunts of Tommy hurt so much that I forgot I was afraid and we fought it out in the fields coming from school. At the end we were both sobbing and I may have won by a very small margin. Later I managed to fight it out with George also. With those two childhood fights came confidence in my physical ability to give and take -- a confidence that saved me from many other boyhood fights. I had learned that a bully thrives on the fears of his victim. Take away the power to frighten and the bully is deflated. 

I learned, however, through a different kind of incident that there are other kinds of fears. For some reason or other my teacher in old 53, I think it was the first grade, locked me in a closet. I was frightened to the point of hysterics. When she opened the door, I was on the floor with my nose to the thin strip of light at the bottom of the door. For many years after, I was afraid of the dark and to this day fear closed places. 

Our new home was lighted with gas continuously by ``pay-as-you-go'' meters. One Sunday evening the lights went out and Pop asked me to go to the cellar and drop a quarter in the meter. To do so I had to go downstairs, through the dark store room and on down to the even darker cellar. I was searching for the slot in the meter when something scrambled away. I ran upstairs screaming and Pop went down and discovered a cat in the cellar. That was another experience that made me afraid of the dark.

Mom was deeply religious and her training was reflected in our early formation of character. All of her social contracts were made at the First Methodist Episcopal Church of Morris Park and her children were drawn into the circle. On Sundays we attended the morning services with her. We could be found at Sunday school in the afternoon and after supper at the evening service with Mom. I can still recall sitting beside Mom and singing such hymns as {\it Jesus Lover of My Soul}, {\it Lead Kindnly Light} and {\it Rock of Ages}. When we were a little older we joined the Epworth League and attended those meetings just before the evening service. Sunday was indeed a crowded day. 

Wednesday was also a special day for Mom. On that evening she went to prayer meeting with a close friend and neighbor, Mrs. Bradford Wicks. Pop rarely attended any of the church services. 

Mom did not believe in any frivolity on Sundays. She would not let us play ``catch'' or any games and it would have taken a real emergency to allow us to ride a trolley car on that day. Sunday was a day we wore our blue serge suits and didn't dare get them soiled. When they became too shiny for Sunday and holiday we used them for school. 

The Methodists were really strict in those days. Each service was an evangelistic meeting in itself with much talk of hell and brimstone. Dancing was frowned upon and drink was a curse. Mom would get real excited when a Catholic neighbor would pass the house with a tin pail on the way to a nearby saloon for a pint of beer. This was called ``rushing the growler'' and the tin pails were large enough to hold far more than a pint. She could not understand either why the Catholics would permit ball playing on Sundays or hold picnics or other festive affairs. 

With no movies, radio, television or automobiles, the church -- no matter what the denomination -- was the social center of the community. Apart from religious services, in the summer there were strawberry festivals on the lawns of the church or private homes, trolley rides to Coney Island and picnics. In the winter there were straw rides in horse drawn sleighs or wagons, or house parties of the various societies.For the children there were two big events. One was the Anniversary Day Parade in early June with ice cream and cake awaiting them in the church basement at the end of the march. With bands playing such tunes as {\it Onward Christian Soldiers} and with the children decked out in their best clothes carrying banners and American flags, it really was a gala event. 

The other big event for children was the Christmas party a few days after Christmas. Following the entertainment, Santa Claus handed boxes of candy and oranges to the children. Sometimes there would be presents from the teachers. One of the then popular G.A.~Henty books, dated December 25, 1902, in my library was a present from my Sunday School teacher, Skidmore Pettit, Jr. 

Kids in that far-away age were no different that kids today, particularly when food was concerned. When refreshments were served on Anniversary Day, there was keen competition on the number of dishes of ice cream we could get away with. The winner was usually the one who could sneak back in line the most times or who would have some friend hand out through a basement window dishes of the stuff purloined from the tables behind the line. 

The same Christmas I received the Henty book, I was presented with a small volume New Testament inscribed ``Presented to Alfred Morris for faithful attendance at the Sunday morning church service by the Sunday School of the First Methodist Episcopal Church of Morris Park L.I.. W.C.~Van Horn, Superintendent. Lincoln H.~Casweel, Pastor. Christmas MCMII.'' 

In the eyes of a nine year old boy, Dr.~Caswell was an old man, but he was probably 35 or 40 years of age at that time. Like his first name, he was a kindly gentleman with a ready smile and made all feel at home. The Sunday School Superintendent, Mr.~Van Horn, also led the singing. His leather-lunged voice is etched in my memory. It literally rattled the rafters and was infectious in its demands on our own vocal chords. He certainly was a top-notch song leader and a fine Christian. 

The preached succeeding Dr.~Caswell as a Dr.~Chadwick, a large, raw-boned elderly man with thick grey hair and bushy sideburns. He hammered home the gospel in a resounding voice. More than any other I have heard, he typified the old-time methodist preacher who believed and preached eternal damnation for the sinner. There were other ministers that followed but non like him. One, in particular, when I was a young man really shook my faith. He deserted his wife and ran away with the voluptuous vocalist of the choir -- the wife of the master of my masonic lodge. It was some time before I returned to the church after that incident. 

Aside from my mother and father, the person whose teachings did most in moulding our moral and religious concepts was our Sunday School teacher for many years, Skidmore Pettit Jr. He started a class of eight to ten year old boys and continued with the group until we were well into manhood. Chester and I were members from the beginning. By words and deeds he instilled in us the urge to be manly, clean and honest. To better hold us together he formed a social club, the C.I.O.C. (Christ Is Our Captain) which alternated in meeting at the homes of the boys. We wore pins inscribed with the letters of the club and were proud to be members of the class and the club. At one time I was president of the club. But this activity, like so many others in life, was not without its tragedy. 

Unfortunately, Mr.~Pettit's teachings evidently did not penetrate the understanding of two of the boys. One became a drunkard and another was indicted for theft. One of his sons, probably the best of the class, was killed in action in the First World War. Most of the boys became respected citizens and many successful business men. 

\begin{center}
------------------------------------------------------
\end{center}
In penning a story of this type, one's thoughts come more quickly than they can be spelled out and account somewhat for apparent digressions. But the object is to present a background to help bring statistics to life. And in order to do so we have to tell something about my youth, business life, courtship and marriage to Mabel, my son, Jim, his wife, Vandy, and our two grandchildren, Scott \& Karen. Then there are my brothers and sister, etc -- and there is more to tell about the early days of the family in Morris Park. 
\begin{center}
------------------------------------------------------
\end{center}

% Timeline: April 1903

Pop continued in the grocery business until the latter part of 1903 and left it with a lot of money owing him and a lot of debts. He returned to work in the railrad shops and continued there until his retirement in 1921. We moved to a duplex house on Sherman Street. It was very much run-down and had no gas or central heating system. We used kerosene lamps and the kitchen range to provide most of the heat. Little portable cylinder-shaped oil stoves took the chill off the upstairs bed rooms but always with the danger of explosion or being tipped over. We were all very unhappy in the old house.

This place was less than a block from the railroad tracks running alongside the railroad shops. In those days the railroad was not electrified and the firing of the steam boilers on the moving locomotives would scatter pieces of coal on the tracks. I can recall walking the tracks with Chester hunting for such coal and half-burnt cinders. We would put them in a bag and drag them home to help feed the kitchen range. We were not asked to do this but we felt it was a contribution that might help Mom put a few extra pennies in the old cracker jar to buy needed shoes and clothing for the family. 

It was experiences such as that that made a deep impression on us and developed the will to strive for success. A will that was further stimulated by avid reading of \href{http://en.wikipedia.org/wiki/Horatio_Alger,_Jr.}{Horatio Alger, Jr.} stories. They were the ``rags to riches and poor but honest'' type of fiction which in my opinion had much to do with building character in the youth of that day. Although considered trite and laughed at today, I believe that, next to our religious training and close family ties, the writings of Alger had a lasting influence on the moral concepts of the ambitions of all who read them.

After a short time in the Sherman Street house, the folks rented a beautiful little one family home around the corner on Chichester Avenue (now 95\textsuperscript{th} Avenue).  It was only a short distance to Public School 57, which we attended. We were very happy in that house. Then we moved to a large house on the corner of Beach Street and Belmont Avenue. This was only a block or so from Adikes farm and I can recall going there after the cows had returned from pasture for pails of fresh milk. Nearby lived a little freckle-faced playmate, \href{http://en.wikipedia.org/wiki/Percy_Crosby}{Percy Crosby}, who would later become a famous cartoonist, the originator of the syndicated comic strip ``Skippy.'' 

% Timeline: 1905

By 1905 or 1906, Pop's finances were evidently in better shape than they had been for many years. He contracted with the local builder (Jeffreys) to build a two family house at 2 Briggs Avenue (now 911 117\textsuperscript{th} Street) just up the avenue from our first home in Morris Park and two doors north of the rear property of the Catholic Church. We occupied the second floor and Mom made a very comfortable home of it. For us children at least, it became the ``old homestead.'' I lived there until my marriage in 1920.

We were all very happy in the new home at 2 Briggs Ave. It was here that we spent many enjoyable years through our teens and into adulthood. Our brother, Kenneth, was born in this house. On the morning of September 19, 1908, Pop asked me to play in the back yard. A short time later, he opened a rear window and shouted ``it's a boy!'' So with the ``kid'' brother added to the family, Mom and Pop were proud parents of one girl and three boys. 

Looking back, Mom must have had a difficult time during her pregnancy. It had been seventeen years since Chester was born. Childlike, we didn't understand her spells of depression and her tears at the slightest provocation. Later, when we were raising our own families we could appreciate the trying period she went through. As later events proved, Kenneth's entry into the family was the source of much happiness and contentment for Mom and Pop. This was particularly true in their declining years with the three other children married and away from home.

Many, many pleasurable childhood memories are associated with the Briggs Avenue home. Playing marbles on a warm spring day under the big elm trees on the Atlantic Avenue side of the open field adjoining our house, jumping on the rear step of a passing horse-drawn ice wagon for a few chips of ice to suck on, climbing and hiding in the trees to puzzle passersby with water from our water pistols, playing ball in the open fields and on weekends sitting on the grandstand of the ball field just across the tracks watching our local heroes ``The Tigers'' play. In the winters there was the usual belly-wopping\footnote{sleigh riding on one's belly, according to {\it British-American Dictionary} (1996) by Catherine M.~McCormick} and playing in the snow. Mom would worry about our always-wet feet and when we returned she made us sit in front of the kitchen range with our feet on the edge of the open oven door. Sometimes we would take off our shoes and put them in the open oven for a quick dry. If we left them there too long the toes would curl up and make it pretty uncomfortable when we put them on again. The smell of wet leather is still reminiscent of those early winter days. 

Then there were the chores -- some not so pleasurable. The new home was heated with a coal burning furnace in the cellar which made a lot of ashes. On Saturday mornings it was our job to sift the ashes for pieces of half-burnt coal that had dropped through the grate. These were used to bank the fire at night. The sifting kicked up a lot of dust and we really needed that weekly Saturday night bath.

I can recall also on each of Pop's pay days taking the mortgage payment book with some money in an envelope to Mr.~Jeffrey's house across from the Methodist Church. When Pop had to work overtime, Mom would prepare a light supper in one of those old workman's dinner pails. I would take it to the shops passing other workmen on their way home. Pop worked on a huge lathe grinding worn engine and car wheels. At times, Pop would be in the midst of a job and could take time only to shout a hearty ``hello!'' I liked that errand particularly when he had the time to chat and tell me about the machines. 

Down the avenue in the old flat building, our first home in Morris Park, the ground floor corner apartment had been converted into a grocery store owned by Henry Voige, a German. Across the avenue on the opposite corner was Dusenberg's candy store where for a penny we could buy a bag full of candy. I can't say it was very wholesome, but what it lacked in quality it made up in quantity. When I was about 12 or 13 years of age I started working on Saturdays at Voige's store.

The Saturday workday was from eight in the morning until eight or nine at night for the magnificent sum of 50$\cent$. And it was hard work delivering big boxes of groceries and bushels of coal also. Many people with only kitchen ranges for heat would buy coal by the bushel in those days. The coal would be dumped in a box behind the range. At the end of the day I would be very tired. I remember one particularly heavy day when I walked home sobbing from exhaustion. I had the feeling that the job was worth more than 50$\cent$. I put the problem to Mr.~Voice, but when he told me that he could get other boys to work for that money, and I didn't have to work for him if I didn't want to, I told him I would be glad to continue as I liked the job. Later, I did get a raise. 

During school vacation the work was not quite so hard but in some respects not to my liking. In the mornings, I would clean the stable and hitch the horse. To a person who has never cleaned a stable by sifting the dirty straw bedding of fresh manure and then raking the straw into a heap for the night's spreading, the experience would be revealing. All that is needed is a pitch fork, a rake and a long-handled shovel, as well as a clothes pin for the nose. The long-handled shovel is needed to scrape up the residue from the sifting, fill it with the stuff and carry it to the manure heap in the yard. One must be sure when leaving the stable to scrape the bottom of the shoes with the sharp end of the shovel. Otherwise the store and the home might capture some of the delightful aroma of the stable. 

Usually I did not mind driving the horse and making deliveries. It was fun and I was a bit proud at being a wage earner. But, when on a Sunday School Anniversary Day, I had to wait on a corner in sight of the paraders and then had to cross the line of march, my ego was really depleted. For these little girls in their holiday frocks and my pals in their blue knickers and white shirts to see me on a wagon in my old duds was the height of embarrassment. 

Friday evenings at the store we would work overtime (but no overtime pay) weighing sugar into 3.5 pound quantities and pouring it into bags until the whole counter was filled. Then we would fold the tops of the bags and pile them for Saturday rush. There were few package goods in those days. Most of the goods were in bulk and weighed on order. Wooden boxes of prunes and apricots were open on one end of the counter. In front of the counter toward the back of the store there was usually an open barrel of sugar, one of mackerel in brine and another of assorted crackers or those ball-like spiced cookies. The open barrel of crackers was real handy when the male neighbors would pause in their shopping for a bit of gossip. Then we would hear some timely cracker-barrel philosophy which to me was the wisdom of the sages. 

Mention of barrels reminds me of election night celebrations. As most bulk goods were packed in barrels, used ones were very plentiful. A week or two before election the boys would start ``collecting'' barrels from the backyards of houses where they were used for rubbish and other building materials, from the yards and other outdoor storage spaces of shops, and from any other places where the owners had perhaps forgotten that it was near election day. At times we would have a collection of 20 or 30 barrels for the big event.

After dark on election night they were taken to some open lot and pyramided as high as possible and ignited. The resulting blaze was our contribution to the celebration of election victors, regardless of party. At our celebrations, one could turn in any direction and see the sky lighted with other such fires of patriotism. Peculiarly, the election celebration of ``borrowing'' and burning of barrels was apparently condoned by our parents for I can remember no time when parental action was taken to stop it. It is true however that the owners of the barrels did not like losing them.

Halloween was another day when the boys stepped a little out of line. For reasons which I cannot understand, the prevailing practice among the kids was to fill the feet of their old long cotton stockings with flour and sock anyone within range of the swinging ``fun-maker.'' As soon as we left school, we would turn our coats inside-out, get our flour filled stockings and star swinging. If one's clothing was not completely white at the end of the day, we were not having fun. Some of the tough kids would be somewhat vicious by using lime or dirt in the stockings and take delight in the punishing blows they could administer. On Halloween evenings everything portable had to be securely anchored. Otherwise the boys would steal gates off fences or take anything moveable and leave them in some neighbor's yard perhaps a quarter mile away. I was unfortunate one night when our crowd started to pull a hearse across the fields and onto a road. I was the one caught. The masked ragamuffin outfits and begging at doors was reserved for Thanksgiving day. That was the day the younger children had their fun.

Such pranks are worth telling if only to show that, even if practice differs, the kids of yesterday are no different than the kids of today. However juvenile delinquency as such is infrequent or at least the expression was seldom used. We were either good boys or bad boys and their seemed to be more good ones than bad ones. In my opinion, this less pronounced delinquency problem was due to more closely-knit family life and more strict parental discipline, closer church ties, tougher police (they used to carry night sticks), and plenty of open spaces for play.

In our area there was ample room for baseball, tennis, and all outdoor sports. There were nearby hills for coasting and tobogganing and ponds for fishing and skating. The growth of population in urban and suburban communities eliminated our playing room and gave rise to gang growth. These groups of boys drawn together by boredom seek an outlet through bullying and crime -- a condition fostered and strengthened in part by the horror of three wars, the portrayal of crime through movies and television, and the decline of the church as a social center of community. 

In our late teens, for example, a group of about eight of us received permission to build two tennis courts on Belmont Avenue. We scraped and rolled the courts, built a wire enclosure, erected a judges stand and player benches and then spent our idle time in strenuous play. I was elected the first president of the Belmont Tennis Club which was in existence for about four years. 

There are many memories of early childhood the recital of which would leave little time and space for the interesting parts of our later family portrait.

\chapter{A New Era}
% Timeline: April 1908

The year 1908 was an eventful one. It opened a new era. As I mentioned before, Kenneth was born in that year. Chester had left school and was working in a textile firm in New York. Vada was in grammar school and I was graduated from grammar school. A sad event was the death of Grandpa LaBonte in Albany where he lived for many years. I had not seen him very much and really knew little about him. My impression is a stocky built man with a large white mustache. He was a conductor for many years on the Boston \& Albany Railroad. Chester and I inherited his prized Columbia Bicycle. Other than makeshifts, it was the first real bicycle we owned. 

In September 1908, I enrolled in Commercial High School on Albany Avenue, Brooklyn (now \href{http://www.brooklynvisualheritage.org/alexander-hamilton-high-school}{Alexander Hamilton High School}). The decision to attend the Brooklyn school was my own. Richmond Hill High offered only a four year academic course. I was raring to get into business and the three year commercial course offered by Commercial High appealed to me. It seemed to offer those subjects that would help me get started more quickly. The course included stenography, typing, bookkeeping, business arithmetic and courses of that type. I could not have taken advantage of this opportunity had it not been for the fact that as a student son of a railroad employee I was entitled to a pass on the Long Island Rail Road. Getting of at the Nostrand Avenue Station, I could walk the mile or so to the school, so there was no additional expense involved. 

But it took me 3.5 years to complete the course. I simply could not master the foreign language subject, German, which I had chosen to take. After I had flunked out in German, I took shop work and came through with average marks, graduating in February 1912. For a short time, I worked as a bus boy during the lunch period in the school cafeteria. During a good part of my high school days, I continued to work at the grocery store on Saturdays and during vacations. 

Those days at old Commercial were happy ones. Maybe I did worry some about my studies and do a lot of cramming to make the grade but those are not the things we remember most. It is true that we can recall that pride of accomplishment that came with some effort we made. However, like in all periods of life, our most vivid memories are concerned with the people we associated with the friends we made. 

Of all the personalities I met at high school, one stands out. He was Wad Smith (Edgar Wadsworth Smith), a schoolmate I met in 1910 when I was 17 and he was 16. Out of this meeting grew a friendship of fifty years. Wad was my closest friend and confidant. Years later we were to become business associates. Wad was a philosopher. He had an analytical mind, a retentive memory and the ability to express his thoughts clearly either orally or in writing. He was an avid reader and an artist of sorts. When we first met, he was Art Editor of Commercial's monthly magazine, The Ledger, and a contributor to its editorial pages. It was through his urging that I contributed a story entitled ``Grit'' which was published.

Wad lived with his parents and a younger sister, Dorothy, on the third floor of a walk-up apartment at 135 Rogers Avenue, Brooklyn. Our friendship ripened through discussions on subjects of mutual interest on our walks together to the railroad station or his home to continue the talks, we found that we had much in common. I had the opportunity to meet his friends from the Bedford Presbyterian Church and he mine from the Methodist Church. But more about that later.

One marked difference between the high school boys (and college boys, too) of that era and those of today was the choice of apparel. Casual dress then was the exception. A suit and stiff white shirt collar was the only acceptable attire. And the collar was detachable from the shirt with matching holes in the front and rear of the collar and shirt neckband for fastening with bone or brass collar buttons. With such an arrangement, one could put a clean collar on a soiled shirt and still look dressed up. Sometimes the very necessary buttons would drop when changing and if they could not be found and no replacements were at hand, we were simply out of luck. To save laundering, a popular collar for the young fellows was made of celluloid. This could be cleaned with a damp cloth either on or off the neck.

Some how or other this prevailing practice on dress made us seem more mature. When we reached the age for discarding knickers for long pants, we were men. At least that is the feeling such a shift gave us. Most of us would keep a sharp crease in those long pants. I can recall just before a date standing in my underwear before an ironing board pressing my only good pair of pants under a damp cloth. Later, when in business and could not get home in time to change, I would go into a tailor shop, take off my pants behind a screen and thrown them to the tailor for pressing. But that was only on rare occasions because pressing by a tailor cost money and money was a scarce commodity. 

Sunday was the day we really put on the dog. I could be seen after Sunday School parading on Jamaica Avenue or in Forest Park in our dark suits, high stiff collars and derby hats (straw hats in the summer, for a fellow wasn't dressed without a hat). Some of us would even wear spats and carry canes. Likewise the girls were out in all their finery -- long dresses and big picture hats and fancy parasols. Later we would congregate in Harsch's Ice Cream Parlor on Jamaica Avenue. If our flirtations had met with success we would treat our new girl friends to big 5$\cent$ ice cream sodas and then walk home with them. On the other hand if we were short of change we would wait until the girls had bought their own sodas before approaching them. Such an approach however was not always successful. 

The usual meeting place for the boys was at the home of the Aston's on Oak Street, the Morrow's on Jefferson Avenue or our home on Briggs Avenue. At the center of attraction at all places was a piano. Both Ed and Tom Aston played and wherever they were their playing would start the boys singing. Neither Bill nor George Morrow played, but their sister Ethel did. At our home Vada played but a difference of four or five years in age is a wide on for teenagers so she was not included in our impromptu sings. A good pal of mine, Art Wenige, played by ear practically any song requested, and he certainly could ``rag'' them. I can see and hear him yet, bounding up the front stairs of our home, making directly for the piano and shaking the room with his ragtime. He and I had a lot of fun composing songs -- he the music and I the lyrics. One, ``A Chapter of Beautiful Dreams'' we tried to have published, but could not break the barriers of Tin-Pan Alley. 

Like all boys, we had our ``bull--sessions'' with serious discussions on topics of the day but particularly on our plans for our future in business. My own ambition was to get into the advertising business. If we had recordings of those discussions they would be mighty interesting to hear today, although I venture to say they would differ little from those carried on currently by present-day youth.

The big mixed social events of our time were house parties with the boys and their girl friends again around a piano or dancing or playing various parlor games -- spin the plate, musical chairs, etc. Then sandwiches, ice cream, cake and coffee and the walk home with the girl friend of the evening. She lived considerable distance from my house, the return from her house to my house through the quiet streets was a lonely one. This was particularly so if the route took me near an isolated wooded area. Then anyone within hailing distance could hear me whistling to keep up my courage. 

It was a lot of fun, also, rehearsing and presenting minstrel shows at the old Arcanum Hall on Jamaica Avenue or the Temple Forum (built much later) on Johnson Ave. This yearly event was a big one which helped us to raise funds through our Knights of King Arthur society (KOKA) for the church.

To the teenager of today, all of this may seem tame. However, one must realize that we had to make our own fun, which we did and had a wonderful time doing so. Radio was not in general use, movies were just being introduced and television was still a long way in the future. The few automobiles were only for the wealthy and not dependable at that.

Outdoor sports were no different than today. We had our baseball team, the Mohawks with a field in Glen Morris, and as mentioned earlier, our tennis club and courts. It may seem odd but those two sports in which skill depended upon speed an keeping your eyes on the ball taught me a lesson which I have never forgotten and was a stimulating force in m business efforts. Years later when I was working desperately in a large corporation to overcome my lack of academic education in competition with college men, I would continually strive to keep my eyes on the ``ball''. In my dreams at night I would hear the impact of the bat on the ball and see the ball soaring into the outfield. I would awaken with the conviction that I could and must keep my eyes on the objectives of the job in hand and really it them hard. By doing that I would have no fear of competition. 

Occasionally Wad smith would bring his baseball team from Brooklyn to play our team. He and I also played a lot of tennis -- a game we played together well into middle life. His father had purchased a small summer bungalow in a new development in Allendale, Staten Island> The developers had built a community club house and a couple of tennis courts. Here we played literally from dawn to dusk on those weekends I was the guest of Wad.

For a couple of weeks one summer, Chester and I and six or seven of the Richmond Hill boys pitched a camp in the woods on the west shore of Lake Ronkonkoma. At that time, Ronkonkoma was very sparsely settled and the lake little used by tourists. It was very picturesque and if a dozen persons were seen on the beach near Hoyt's dock we would have considered it crowded. Another year I spent a vacation of a week or so at Comp Wopowog in Hampton, Connecticut. With Wad's crowd I also roughed it for a short time at a cabin on Cedar Lake near Denville, New Jersey. The cabin was loaned to us by the father of one of Wad's friends. 

We also had some good times at a large club house built on stilts at Goose Creek on the Rockaway Beach line of the Long Island Railroad. We would go there at the invitation of Harry Maass who was one of the Methodist crowd. Harry was an orphan and for very little pay while at school he would work on weekends cleaning up at the club. During the week, he was permitted to use the house. And he did in a big way by inviting our crowd to spend the night with him. The club had six or eight bedrooms all equipped so we were taken care of . I can recall the swims and sometime after midnight playing cards. The real thrill was diving from the tower at the end of the dock in the pitch blackness of night. 

All of the incidents and experiences mentioned so far occurred before our entry into World War I in 1917, some of them while I was in high school and some after I had started to work.

\chapter{Office Work}

My first experience in office work was with the \href{http://en.wikipedia.org/wiki/Frank_Munsey}{Frank A. Munsey Company} in \href{http://en.wikipedia.org/wiki/Flatiron_Building}{the Flatiron Building} in New York. It must have been in 1910 or 1911 during summer vacation that one of the boys suggested that a friend of his was going on two-week vacation and would appreciate my substituting for him as office boy. I did and was out of the office more than in because of the amount of messenger work I was assigned. As Munsey was one of the largest pulp magazine publishers at that time I received good bonuses in the form of copies of their magazines. I don't remember what pay I got. 

% Timeline: April 1912

Following graduation from high school in February 1912, I took my first job with the United States Title Guaranty Company in Jamaica. Frank DeBevoise, a fellow member of our church was employed by the company and asked me if I would be interested in coming with him. His older brother James was an assistant secretary of the company. I gladly accepted the job and was employed in the dual capacity of office boy and title searcher at a weekly salary of eight dollars. This was it. It was the beginning of a business career. I was happy. I could give mom five dollars a week toward the family income and still have three dollars a week for myself.

I did odd jobs about the office, helped with the photo-stating equipment and searched records in those very large canvas-backed alphabetically indexed books in the nearby county clerk's office. I enjoyed the work, my associates and the environment of a busy office. That is until I was tempted by a better offer by Harry Elsebough, our next door neighbor on Briggs Avenue. He was a bookkeeper and credit man for Walter W. DeBevoise (no relation to the DeBovoises in the Title Company), a candy manufacturer in 163 Carlton Avenue, Brooklyn. His offer of a job as assistant bookkeeper at ten dollars a week was just too tempting to resist. I accepted, left the Title Company on August 31 and on September 16, 1912 started my new job with DeBovoise.

On that morning I took a Long Island Railroad train to Flatbush Avenue and walked the mile or so to the factory and reported for work at 7:45 am. The factory was a three-story brick building about 200 feet by 150 feet. After the fairly modern office of the Title Company, the DeBevoise office was really antique. A better word might be old-fashioned. The office consisted of an eight foot long, two-sided bookkeeper's desk at which one could stand to work or sit on a high stool. A shelf elevated over the length of the desk was for ledgers or papers. At one side stood Harry Elsebough, coatless and with the green eye-shade on his forehead. The other side opposite Harry and behind me was a little corner sink with a mirror above it. 

Backed up against the front end of the ``desk'' was an old rolled-top desk for use of the city salesman. Back of Harry and near the front door was the combination typewriter and bookkeeping machine operated by a middle-aged woman. In one corner of the room was a walled-in cloak room and in an opposite corner nearest Harry and the typist, about the same size as the cloak room, was Mr.~DeBevoise's private office. This consisted of a large roll-top desk, a big leather-covered swivel chair, another cane-bottom chair, a wash bowl and a small wardrobe closet. The window ledge was used for filling odds and ends and for candy samples the boss was interested in. Add to this first impression the (at first) overpowering smell of chocolate and roasting peanuts and you have my introduction to the business in which I spent many years.

Here I was in my second unplanned job and one quite different than the first. Neither, however, were in the advertising field where I thought I belonged. I didn't get into that field for the simple reason that I didn't try to do so. Like so many boys of that era -- and of today, too -- I was opportunistic and took the first job offered because it provided an immediate income. Perhaps if my parents had the financial means to send me to college or I had the guts to make that opportunity myself, I would have been more selective in seeking a job and waiting until I got what I thought I wanted. As it turned out, however, the job with DeBevoise gave me a most liberal education in practical business and a broad background for my third and last job in business which was to come 17 years later.

Walter W.~DeBevoise was the sole owner of the business and a very sharp operator. A calm, dignified man of about 50 with graying black hair, he evoked the like of a \href{http://en.wikipedia.org/wiki/Parson}{parson} rather than a successful business man. Two relatives worked with him. Gus, a brother, was superintendent and Clarence, a nephew, his assistant. Gus was gruff and could be very tough with the workers. Clarence, on the other hand, was talkative, witty and popular with his associates. Harry Elsebough, my immediate boss, who faced me across that long desk was a tall handsome man with a bushy brown mustache. He had a greater tendency to tell stories and play practical jokes than to get down to business. All-in-all it was a pleasant place to work despite the long ours and hard work.

The work day started at 8 am and ended at 5:30 pm. Saturday was ``half-day'' from 8 to 3:30. My duties, as outlined on that morning of September 16, 1912, were to dust the office furniture with one of those long feather dusters, open the safe and place the ledgers on the long shelf above our desk and then start posting the journal entry of the previous day's deliveries. I was also to help the girl type the billings. As it developed, however, in the day-to-day activities, I would do as Harry and the boss would do, that is pitch in on any job in the factory that required help, particularly in the shipping room tying boxes and helping load the trucks. 

The boss would arrive at the factory at about 8:15 am. He would first open the mail piled on his desk being very careful to cut the envelopes cleanly. He would pile the clean sides of the envelopes on his desk for scratch paper. Then with a few of these and a stub of a pencil he would tour the plant. When he returned he had the ``cost accounting'' data noted on the clean underside of the clipped envelopes. This data included the number of points of material in each batch of candy, the number of people on each operation, the time required to complete the jobs and all other facts gathered at first hand that would help him in estimating costs. He did not have to refer to invoices for the costs of materials. He knew just what they were. Despite this apparently inefficient method, the business prospered in those early years.

DeBevoise specialized in penny candies, the kind I used to buy in Dusenberg's store when I was a kid and were very popular because you got a lot for the money. Big pieces of chocolate-covered marshmallow, colored coconut-covered marshmallow bars, chocolate covered marshmallow pretzels, drops and the leader, Uncle Same Bars, made of ground peanuts and cheap chocolate. These bars were about the size of 5$\cent$ bars today. Then there were such novelties as Fried Eggs, little tin saucepans filled with white cream and topped with a piece of pink cream to resemble the yolk. A little tin spoon went with this ``work of art'' and the kid got it all for a penny. DeBevoise bulk candies included jelly beans, coconut bon-bons, chocolate cream drops, and the real treat of them all, hand-dipped chocolate almonds.

The candy was sold to jobbers who in turn sold to the little retail stores. There were many of these little stores. What few chain food stores in existence at that time did not sell candy and few of the chain tobacco shops did. The ice cream and confectionary stores to a large extent handled packaged and bulk candies. In retrospect, one wonders how the small candy stores catering to children met their expenses. For example, DeBevoise's Uncle Sam Bars sold to the jobber in boxes of 48 pieces for 35$\cent$. He would sell it for 38 or 40$\cent$ to the retailer who would have to make 48 separate sales for a return of 8 or 10$\cent$. A little greater return was possible for both the jobber and retailer with other candies packed 72 or 100 in boxes selling for 40 and 50$\cent$ to the jobber. In light of the narrow margin of profit in the candy business, maybe my starting salary of ten dollars a week was not so small after all.

Girls in the factory averaged about eight dollars and men about 10 dollars a week. The foreman and skilled candy makers got more, of course. Because of absences and off-season lay-offs, many of the pay envelopes -- which I helped to stuff -- contained only three or four dollars. 

Most of the girls worked in heavy sweaters in the refrigerator rooms packing chocolate candies after they were sufficiently chilled and hardened. Outside these cold rooms men lined up on each side of slow-moving endless belts and placed uncovered candies on the belts which ran through the \href{http://en.wikipedia.org/wiki/Enrober}{enrober} where they were covered with chocolate and moved on into the cold rooms.

In addition there were separate departments for the manufacture of chocolate, coconut goods, and for ``panned goods'' such as jelly beans and jawbreakers which were coated and polished in large revolving kettles. 

During the early period of my associate with DeBevoise, sanitary and working conditions were not what they are today. Food factories were far from clean, and pure food laws either non-existent or not enforced. Highly flavored and dark colored candies covered a multitude of sanitary and production sins. Working conditions were equally bad. Management was int he saddle and rode hard on labor. One example in our plant can be cited. Two men feeding an enrober quarreled and slowed down the line. In a matter of seconds, the superintendent and the foreman manhandled these workers and had them back on the line without another murmur.

Twenty years later the pendulum had swung widely in the other direction. Labor was organized and began its usurpation of power at the expense of management and the progress of industry. At this writing, labor is still in the ascendency and management greatly handicapped in efficient operations. For the future of industry and the economic progress of America, it is my hope that an equitable balance between management and labor can be reached. 

After three or four years, I began assisting in sales. I was given the South Brooklyn territory in which there were about a dozen jobbers. These were covered by trolley car and on foot in one afternoon a week. I also helped out in the Yorktown section of Manhattan on Fridays and filled in for Bill Kimberly, our city salesman, in other sections when he was ill. Another additional job, one much to my liking, was writing advertising copy for two of the trade magazines. These were but quarter page ads but gave me the opportunity to see my copy in print. Then on my 24\textsuperscript{th} birthday, April 6, 1917, the United States entered World War I.

% Timeline: 1917

Meanwhile, home in Richmond Hill our social activities were humming along as outlined in a previous part of this story. Wad Smith had taken a job with the Brooklyn Daily Eagle and we were together quite a little in Brooklyn and in Richmond Hill. I was as much at home with his friends as he was with mine. So between the two groups I was kept busy -- socially at least. As our apartment on Briggs Avenue was a bit crowded, Pop and I partitioned off two rooms in the attic with plaster board. They were fairly large rooms but bitter cold in the winter as it was not piped for heat. I would have to undress in the kitchen on these cold nights and hustle into bed under a half-ton of blankets. Later, when Grandma LaBonte came to live with us, she had the other room. She was ill and senile at times. On several occasions during the night she would fall out of bed and I would grope through to her room and lift her back into bed. Mother had a difficult time as Grandma Morris, also ill, lived with us before Grandma LaBonte came. Grandma Morris died in 1917 two days before her brother, David Swayze. Grandma LaBonte died in 1919. 

Chester had been keeping company with a little auburn haired girl Lizetta Letts for two or three years. Lizetta lived down the avenue diagonally across from our first home in Morris Park. After Chester met Lizetta, no other girl meant anything to him. They were married in 1915 in her home. It was quite a wedding. Most of our and their relatives were there. I am thinking particularly of Carrie Mather, mother's sister, her husband Raymond and their two children, Fred who was a little older than Chester and Effie, a beautiful young woman, who was about Chester's age. Fred would visit us occasionally but Effie and her father we lost track of. Uncle Raymond was a heavy drinker and years later Aunt Carrie left him and lived with the folks in Ronkonkoma as a result of that broken home. 

Perhaps because of the distance separating us, we rarely saw Mom's family. Mom had a brother, Alvie LaBonte, in Rensselaer whom I had met only once. That was when Clarence DeBevoise, Louis Pesagno, our chocolate foreman at the factory, and I spent the a weekend sightseeing in Albany. I had heard that Uncle Alvie worked with a paper manufacturing company so found his business address and dropped in for a five minute chat with him. I never met his wife or children and do not even know their names.

In 1917 Vada married Roy Rubb, a tall slim good-looking boy who lived in Richmond Hill. As Roy was not in our group of boyhood friends, I had not met him until he began courting Vada.

\chapter{The First World War}

After the declaration of war in 1917, the boys of my age were confused. We had lived in a peaceful world most of our lives. The \href{http://en.wikipedia.org/wiki/Spanish-American_War}{Spanish-American War} in 1898 occurred when we were babies and the only other military disturbance was the expedition of the United States sent into Mexico in 1916 in an attempt to capture the Mexican Bandit \href{http://en.wikipedia.org/wiki/Pancho_Villa}{Francisco ``Pancho'' Villa}, who had raided our borders. In comparison with the war that had been going on in Europe since the summer of 1914, these were mere skirmishes. Our patriotism had been fanned to a fever heat and we wanted to offer our services voluntarily. We did not want to be drafted. We wanted to choose the branch of service we thought we could be most helpful. We certainly were confused. 

In this period of indecision, I can recall the urge to toss a coin and accept its fall on joining the army or navy. One day my business took me in the vicinity of Battery Park, Manhattan. It was a bright, warm day and I stood on the waterfront gazing at the shipping in the harbor, I made up my mind that the navy was where I belonged. 

I enlisted in the Naval Reserve as a seaman third class and was called to duty as a yeoman third class and assigned to Headquarters, Third Naval District at 280 Broadway, New York City. After all the pondering and indecision, here I was behind a desk while many of my friends were in the more romantic ends of the service -- at least, I thought then that they were romantic. Later, I realized that it was fate or good fortune that kept me out of combat, but at that age pride was a factor that influenced a feeling that we were slackers hiding behind a desk.

My duties were entirely clerical. The uniform and the discipline were the only things different from civilian life. I lived at home and commuted to New York like a civilian. I don't think I looked particularly heroic in the bell-bottom pants and gob's jacket and I was very sensitive about wearing them. Even though I was living at home, it was against regulations to don civilian clothes while there or any place. During the war the uniform was mandatory at all times. By the end of 1917, I was a yeoman first class and wore three stripes on my jacket sleeve and in May 1918 was made a Chief Yeoman and proudly wore the uniform of a chief petty officer. 

In January 1918, I learned that examinations shortly would be held for a paymaster training course at the \href{http://en.wikipedia.org/wiki/United_States_Naval_Academy}{Naval Academy in Annapolis, Maryland} starting on August 1. This seemed to offer a real opportunity to get away from New York and perhaps get a commission and an assignment to sea duty. With the recommendation of my commanding officer and letters from DeBevoise, United States Title and the principal of Commercial High, I applied for permission to take the examination. Permission was granted and I took and passed the examination.

Much to my disappointment, I was not included in the Annapolis group. I did so much want the opportunity to take training in that historic academy. But my disappointment was tempered with the announcement a short time later that another training course, United States Navy Officer Material School for the Pay Corps, was to be established at Princeton University and that I would be included. Now I was happy. I was about the remove what I thought was the stigma of desk duty in the City and, at the same time had proven to myself that I could compete successfully in an effort to improve my status. It was the first real test of competitive competence outside of school and sports. My folks were proud. And I hoped a certain charming young lady I had met a short time before shared that pride. 

In the spring on 1918, Art Wenige and I dropped in at a dance for servicemen (Art was an Army trainee at \href{http://en.wikipedia.org/wiki/Camp_Upton}{Camp Upton}) given by the USO in Brooklyn. There Art met a young lady in a Red Cross Nurse's uniform. When Art suggested a date, she agreed on the condition that a friend of hers come along if I would care to make it a double date. I agreed but wondered who was being imposed on me as I had had some experience on double dates and never seemed to pick a winner. 

We kept the date. I can't remember where we went or what we did because I was in a daze. I had met THE girl -- Mabel Norseen, an 18 year old slim blond girl with a ``peaches and cream'' complexion. She lived at 563 10\textsuperscript{th} Street in the Park Slope section of Brooklyn and I can remember leaving her that night after she promised me another date and somehow or other finding my way home to Richmond Hill. I was really ``hit hard.''

I wondered what she thought about me, whether she felt that I was the ``boy of her dreams.'' I doubt that she did because I was far from a romantic figure in that gob's uniform which accentuated my thin, pale face and my slightly bowed legs. These are the thoughts that bothered me after that first meeting. But she had agreed to another date so maybe she did like me. Shortly after that eventful first meeting, I proudly called on her wearing my chief petty officer's uniform which gave me a lot more confidence in my burning desire to make a favorable impression. Maybe I did because we were together several times before I left for the paymaster's school at Princeton with her promise to write after. 

Along with 100 or more hopefuls (perhaps 200), I reported to the Naval officer in charge of training at Princeton early in September 1918. We were billeted in beautiful Graduate College at the far end of the Princeton campus and I was thrilled with the place. At long last I was at ``college'' or at least in the environment of a college. I was billeted with three other ``M's'' -- Murdock, Moses and Morrisey -- in a room in the graduate building. It was an oddly assorted group. Murdock was a Harvard man, later to become a professor there; Morrisey, a tough little Irishman from New York City; and Moses, a quiet spoken boy of Jewish descent and also from the City. 

The training was not too difficult. It consisted of lectures on navigation and on subjects concerned with naval regulations and paymaster's duties with on period a day devoted to actual military training on the parade grounds. I cannot understand why we took such military training when our jobs were to be on paper work. It must have been for the discipline factors. 

I so much wanted Mabel to come to Princeton for a weekend so I could show her the University and how we lived and studied at the College. But most of all I wanted to see her. We were corresponding regularly and had made tentative arrangements for her visit. However, her parents were very strict and she was not permitted to come.

An army officer training school, also at the University, had formed a football team and challenged the navy school. I tried to make the navy team but after a few bruising practice sessions realized that a skinny 135 pounder didn't have a chance in competition with the heavy-handed huskies on the squad. When we finally played the army I was happy that I wasn't in the game for it was a bruising battle-royal with no holds barred. The army won. They had the least number of casualties. 

A day long remembered was November 7, 1918 when word was flashed (erroneously) that the war had ended and an armistice effected. A group of us were on leave in New York City and became a part of the wildest, most tumultuous demonstration I have ever witnessed. We were in the vicinity of Times Square. Shouting, cheering and weeping crowds jammed the streets. Anyone in uniform, regardless of whether they wore service stripes, was literally mobbed. Those men in our group were embarrassed by the homage accorded us, which, incidentally included indiscriminate hugging and kissing by the young gals and by the old ones too. We were being pushed into the Hotel Astor and found ourselves at a table in the roof garden in the noisiest dining room I have ever been in.

Men in uniform could not be served drinks in a public place, but plenty were passed under the table to us by civilians at other tables. It was a night long to be remembered. It was a night which rightfully belonged to the millions of service men on the battlefields and on our fighting ships. There was a bad letdown the next day when it was reported that the rumor was false and there had not been an armistice declared. That was to come four days later on November 11. But the premature celebration was nevertheless the big one. 

In connection with the passing of drinks to service men on the night of the big celebration, I might mention that I had never tasted alcoholic beverages of any kind until I put on a uniform. Pop and Mom were teetotalers and drink in our home was not even thought of. On those prewar occasions when I was with friends in a cafe, I would order soft drinks. But when in uniform, I guess I wanted to show my independence. 

Back at Princeton we were winding up our training and taking examinations. We were commissioned on November 15 and ordered to the Naval Training Camp at Pelham Bay, New York. The final night at Princeton was a wild on. Someone started a snake dance and before it ended some 200 men were in line snaking through the rooms over the tops of anything in our way and then on out to the campus. We never did hear what action was taken on replacing the broken furniture. 

At the Pelham camp we were given time to obtain our uniforms. I went to a custom tailor (Haines Brothers) on Fulton Street near DeKalb Avenue in Brooklyn and ordered a uniform, overcoat, cap and believe it or not, a sword. Regulations at that time called for a sword as part of an officer's equipment. What we were to do with it was a big question. All of this equipment was at our own expense. 

Ten days later, resplendent in my new Ensign uniform, I reported to the receiving ship in Brooklyn (barracks on the short of New York Harbor) to await orders for sea duty. These orders did not come until nearly three months had passed. As a staff officer, my duties were quite limited during this waiting period. Once or twice a week when officer of the day, I slept at the base. I was on a subsistence allowance which meant that when not on night duty, I was given an allowance for board and permitted to sleep at home. It may be of interest to note that this allowance plus my pay as an Ensign totaled more than my earnings at DeBevoise. Because of limited facilities at the base, the officer's mess was at the nearby \href{http://en.wikipedia.org/wiki/Crescent_Athletic_Club_House}{Crescent Club} which gave me an added thrill of lunching at one of Brooklyn's finest clubs. It was quite an experience for a poor buoy and I was quite proud I will admit. 

But my biggest break of luck in being stationed in Brooklyn was the opportunity to see Mabel frequently. We were ``keeping company.'' I rather think she took some pride in introducing her officer sweetheart to her friends. I met the members of her sorority and those of her boy friends not in service abroad. I seemed to be accepted as one of the crowd. I had little time for those of my old pals who were still at home. 

Wad and several of the old Brooklyn group were in the army or navy abroad. One of his close friends, Al White, whom I had come to know very well, was killed on the battle fields. Fred Pettit, the son of our old Sunday School teacher and a charter member of our class, was reported missing in action and never found. Other friends were in the service in various parts of the world. 

A short time before the war, Wad had taken a job as a secretary with J.P.~Morgan \& Company on Wall Street. He was enlisted in the 305\textsuperscript{th} Infantry training at \href{http://en.wikipedia.org/wiki/Camp_Upton}{Camp Upton} on Long Island. After receiving basic training, he was assigned to the staff of the Second Assistant Secretary of War in Washington as a Second Lieutenant. Now in Paris with this staff he was in the type of work in which he was so well qualified. He was advanced to a First Lieutenant and then to Captain. In France, he became friendly with another member of the staff, Merle C.~Hale. After the war Merle and I became close friends -- a friendship that ripened as the years passed and was terminated by his death in 1961. 

On February 27, I received orders to report to the \href{http://en.wikipedia.org/wiki/USAT_Buford}{United States transport Buford} upon arrival in port. I checked daily but had no word of its arrival. She was long overdue. On March 9, I read the following Universal Service dispatch in the press:

% indent this
\begin{quote}
``Newport News, March 9 -- After floundering in heavy seas for twenty-one days, losing her course many times in a terrific storm, the United Stats transport Buford, with 1025 returning soldiers on board, came into port to-day. She was towed by rescue tugs that picked her up off Cape Henry late yesterday. 

Fuel was almost exhausted and the steering gear gone when an S.O.S.~was picked up by tugs that rescued her after a terrific battle with the waves.''
% indent this
\end{quote}

I wasn't too pleased with this news and wondered what was ahead of me. I took off for my assignment the next day and as it was late in the evening when I arrived I checked in at a hotel. The next morning I decided to get my bearings on the hereabouts of the ship before reporting. I discovered that the Buford was an old merchant ship taken over by the Navy as an Army transport. Alongside the trim naval ships at dock, she ever did look shabby. With the takeover of the ship by the Navy, the civilian officers had been commissioned as Naval officers. 

It was fortunate that I obtained this information before reporting for duty because if I had reported as naval regulations required with that shiny new sword at my side I think I would have been tossed overboard. I went back to the hotel, packed my gear and walked to the ship with it under my arms. I simply introduced myself to the skipper, Commander Olsen, and told him I was reporting for duty. If I had reported to that old sea dog according to regulations, I hate to think of the consequences.

Fortunately for me, a raw Naval Reserve Supply Officer, the Chief Yeoman assigned to me was a regular Navy man with many years of experience in supply practice and regulations. Like all seasoned chief petty officers he knew his job and was a fine chap. He did the work and I was more or less a figure head treading softly for fear of making myself a conspicuous neophyte. The first test of such fear came when I had to climb down a rope ladder over the side of the ship to a supply barge below. It was night and that swaying ladder seemed a mile long. With gobs watching, I managed to scramble down that later, check the supplies and climb up again without, I hoped, too much loss of dignity. 

The first night on the Buford I joined a group of officers and a civilian salesman for a game of poker at a hotel. One of them produced some liquid refreshment which resulted in more of a drinking bout then a card game. The civilian got a bit out of control and two of the officers were forcibly restrained by the others from throwing him out the window. So ended another episode in my efforts to be accepted as a regular guy by those ex-merchant sailors. 

A few days later the Buford took off for France but didn't get much beyond the harbor before there was more propeller trouble and she was forced to return to the dock. Thus ended my military duty for two weeks later I was detached from active duty and returned home on March 22, 1919. Like many, many other volunteers, we had no battle scars or had won honors in fighting. But we had gained a lot of experience which I think was helpful in the battle for business survival and progress in the years ahead. Back in civilian life I was eager to resume my old job with DeBevoise. 

\chapter{Back To Civilian Life}

The transition to civilian life was a bit rough. For the past six months my commission had carried some prestige and now, so to speak, I was at the bottom of the heap again. As I needed new suits -- but more likely because I wanted to show-off -- I wore my uniform during my first week or so on the job. I must have looked silly calling on jobbers in the South Brooklyn territory wearing an ensign's uniform and lugging a big sample case or standing behind that big desk in the office posting the journals. But the uniform was soon discarded and I eased into the business grooves again.

DeBevoise had been enjoying good business during the war despite material shortages and sales were running over a million dollars a year. He was now granting monthly bonuses to the foremen and key personnel and informed me that I would receive one also. Shortly after I was given additional sales opportunities along with a used \href{http://en.wikipedia.org/wiki/Ford_Model_T}{Ford Model T}. Although it was understood that the Ford would be garaged in DeBevois's new warehouse and garage across the street from the factory, I did drive it home on those Saturdays I worked the Long Island territory and would hold it over the weekend. It was on such a Saturday that I invited Mabel to join me and we had a very nice trip together, that is until the time I thought I could look into her eyes and at the same time weave through traffic. I found to my dismay that I did not have such dual vision when I drove the Ford into the rear of a parked car on a street in Freeport with a crash heard a block away. Fortunately the car I hit was a heavy one with a strong bumper and not damaged at all. The Ford, however, was damaged severely and had to be towed to a garage for repairs while we went home on a Long Island train. The owner of the parked car evidently sense the situation and did not make any trouble for us. We thought he was a great guy. 

% Timeline: 1919

The summer of 1919 was indeed a happy one. The uncertainties of the war days were behind us and I had what I thought was a good job with good prospects. But the crowning happiness was the love Mabel and I shared. We would see each other one or two nights a week and on Sundays. If we had lived near each other I would have seen her every day. But the problem was transportation. If we were at a late party and I missed the last night train on the Long Island Railroad (1:10 am) there was a long trip via elevated railroad with a long walk on the Richmond Hill end. 

Mabel had graduated from the Brooklyn Training School for Teachers and was teaching in the Park Slope section. On Wednesdays, which were the days I worked that territory, I would arrange my schedule so that I could meet her and drive her home, the long way of course. Many Sundays we spent on the beaches with Jack Glanders and Ida Hansen or with Harry Maass and his bride-to-be, Billie. 

In the early winter of 1919, Mabel and I thought that we should obtain Dad Norseen's blessings on our intentions (although there wasn't much doubt that he and Mother Norseen had not anticipated them). So after much discussion in the parlor of her home one evening, Mabel called Dad Norseen in, told him I wanted to talk with him and left the room. I can recall that we both stood as we talked. Neither of us smiled as I stammered through a monologue of my love for Mabel and my expression of confidence in my ability to take care of her on my salary of sixty dollars a week and the good prospects ahead for me in the candy business. I also mentioned that I had a little bank account, but did not mention how little. I cannot remember whether he said ``OK' or just grunted, but that experience was over and we were officially engaged. 

On Christmas 1919, I had the pleasure of placing an engagement ring on Mabel's finger. Arrangements were made for a wedding in October 1920. Our happiness was marred, however, by the untimely death of Mabel's 18 year old sister, Florence in January 1920 and the attack of influenza which Mabel contracted at the funeral service at the cemetery was to have serious repercussions in the first two years of our marriage. 

Meanwhile, Wad Smith had returned from his military service in Paris as a Captain and with a French bride, Marthe. They had met in Paris and had been married after a brief courtship with Wad's good friend Merle Hale, also a Captain, as best man. Marthe, a widow with a baby boy, Jack, had some difficulty with English on her arrival. We had a lot of fun with her and Wad and saw them frequently. Marthe, like so many other French and Europeans had the idea that all Americans were wealthy. As she told us later after Wad was really ``in the money'' she had her eyes opened when Wad took her to his parents' little flat and the simple bungalow they owned in Staten Island. 

Wad had accepted a job as a secretary with the General Motors Export Company and suggested that I too join up with that progressive little company. I felt, however, that the DeBevoise outfit offered better opportunities. Mr.~DeBevoise had promised that eventually he would turn the business over to his key people. I still do not know whether or not I made the right decision. Ten years later I did make a connection with General Motors but meanwhile Wad had made an outstanding success with GM. During 1920, for of reasons I do not now recall we lost frequent contact with Wad and Marthe. I think during that time he spent some time in Europe for GM.

Mabel and I were tied up with our own problems as mentioned previously and with our plans for our wedding. The big event took place at her home on 10\textsuperscript{th} Street on October 20. It was a simple little affair with only the immediate family in present -- Mom and Pop Morris, Mother and Dad Norseen, Chester and Lizetta, Vada and Roy, and Aunt Carrie Mather. Jack Glander was my best man and Hylda Booth was Mabel's Bridesmaid. Dr.~Young, Pastor of Mabel's church officiated. Jack, who had been taking vocal lessons and had a very fine tenor voice, sang \href{http://en.wikipedia.org/wiki/Oh_Promise_Me}{``Oh Promise Me.''}\footnote{Music by Reginald De Koven and lyrics by Clement Scott.} Mother Norseen served the wedding supper. It was a quiet affair but exciting nevertheless. In order to escape the rice and kidding, Mabel rode the \href{http://en.wikipedia.org/wiki/Dumbwaiter_(elevator)}{little kitchen dumbwaiter} into the basement and ran out through the front basement door to the car where I was waiting, but did not succeed in evading our well-wishers. 

Our stop for the night was the old \href{http://en.wikipedia.org/wiki/Hotel_Pennsylvania}{Hotel Pennsylvania} in New York City. Here I experienced the embarrassment of signing ``Mr. \& Mrs.'' for the first time while rice fell from our clothing. The next morning before getting up we discovered a couple of painters peering through the transom. When we left we saw that the scaffolds they had used for redecorating the halls. This served as an entertaining conversational piece with our many other newly married friends. Our next stop was a little hotel at Delaware Water Gap. Not very far from home, not very fancy and not expensive. But it was all we could afford and it seemed very wonderful to us. That October in 1920 was \href{http://en.wikipedia.org/wiki/Indian_summer}{Indian Summer} with a vengeance. We had expected crisp cool weather and had dressed accordingly. Our attempts at mountain climbing and hiking therefore were not too comfortable. Here too we found that the influenza Mabel had suffered had taken its toll as she found it difficult to climb or take long walks.

Mabel had planned to continue teaching. Her folks wanted her to capitalize on their investment in her education and to build a backlog of experience in teaching as a security safeguard in the event she needed it in future years. For that reason we had decided to live with the folks in their 10\textsuperscript{th} Street home while she continued with her teaching. We purchased furniture for one of the two bedrooms in the ``railroad'' flat and boarded with the folks on returning from our honeymoon. With board of only ten dollars a week and our combined salaries we figured we could build a little nest egg. But that thought was soon forgotten in the worry over Mabel's health. 

Her condition worsened rapidly and before the end of 1920 examinations by her family physician and a lung specialist indicated that she had tuberculosis. In January 1921, a little more than two months after our marriage, we arranged for her admission into Loomis Sanatorium in Liberty, New York. Despite rest and excellent medical attention at Loomis, her condition deteriorated. In the following June, I had a noted New York City specialist Dr.~Roy Upham, visit her at Loomis. His opinion was negative. He felt that she could not survive another two or three months. However, neither he nor the good doctors at the sanatorium had taken into account her will to live. This fighting spirit plus the artificial collapsing of the diseased lung (a new method at the time) brought a hope that was realized in 1922 about two years after our marriage. 

Meanwhile I continued to live with the Norseens rather than back in Richmond Hill with the thought that my presence might help alleviate their sorrow. For us it was a long period of alternating hope and despair. For me, in addition, it was a long period of physical as well as emotional strain. Each weekend regardless of weather, I would drive to Liberty over Rout 17 which at the time was a narrow winding road with long grades and extremely dangerous in wet weather. The old Ford had been turned in on a Maxwell roadster with side curtains which could be rolled up in sunny weather ( -- incidentally, I had to purchase this car. DeBevoise contributed only the trade-in value of the Ford). After a few months of weekend driving alone, I teamed up with Ted Lauer who was the husband of a fellow patient of Mabel at Loomis and would alternate in the use of our cars. He lived in Hackensack and many times I would spend Sunday night at his home rather than continue on to Brooklyn after a particularly difficult drive. Ted worked in the office of \href{http://en.wikipedia.org/wiki/E._F._Hutton_\%26_Co.}{E.~F.~Hutton} and in later years became a partner in that prosperous firm of stockbrokers. Unfortunately, Ted's wife died shortly after her return from Loomis. 

I do not want to dwell for too long on this Loomis episode in our lives. It is one we had to put out of our minds. It is strange that the things most of us recall are the pleasant ones and rarely think of these which have been difficult. However, I would like to mention two things that helped to soften the impact of our concern and worry during Mabel's illness. One was the overpowering optimism and cheerfulness of the patients at Loomis, many of whom had little hope of recovery. The other was the good cheer and fellowship of the husbands and relatives who were accommodated at a little boarding house on the hill at the back of the Sanatorium. Here we had our meals and comfortable beds after our long trips and short visits with our wives at the hospital. Here, too, we had interesting chats with our new friends.

It was a great day when I was told that Mabel would soon be discharged -- an outstanding example of a ``cure.'' One condition of her discharge was to live in as high and dry a climate as possible within a reasonable commuting distance from my office. After a long search for such a location, I answered an advertisement for boarders at a West Summit, New Jersey residence. It was a beautiful old home with spacious lawns and many shade trees. It was an answer to our dream. The owners, the Steidel's, a couple with three teenage children, had suffered temporary financial reverses and welcomed us to their home where we lived as members of the family for about a year. Mabel had no household obligations and plenty of opportunity for sunshine and rest. 

One hardship for me, but one I cheerfully accepted, was transportation to and from my office in Brooklyn. This included the bus from West Summit to the Summit Station, the Lackawanna train to Hoboken, the Hudson Tube to New York, the subway to Borough Hall Brooklyn and the trolley to the office. Five modes of transportation and at considerable expense. 

After a year, Mabel was able to take up housekeeping in a new apartment house at 1020 President Street near Prospect Park in Brooklyn, and within a few moments ride on the Vanderbilt Avenue trolley to my office. Here we furnished our three-room ``dream house'' and were back to normal and beginning the kind of life we had hoped to begin three years previously. Here we had the opportunity to renew friendships and family ties. 

Shortly after our return, Mabel's folks sold their 10\textsuperscript{th} Street house and moved to one on 8\textsuperscript{th} Street. Meanwhile, Pop and Mom Morris had built a small bungalow on the old ancestral property in Ronkonkoma which Pop had inherited from Grandma Morris. This was completed in 1921 on a tract of land on the north side of the track. However, Mom and Pop did not move there until a year or so later. Chester had lost his job with a textile organization and moved into the new house with his little family for the period of job relocation. He and Lizetta now had two children -- Evelyn born in 1916 and Roger born in 1920. In 1922, Pop and Mom sold their home in Richmond Hill and moved to their new home. From there, Pop commuted to his work in the Morris Park railroad shops until his retirement in 1936.

But despite the long hours of travel, Pop was happy to be back in the area of his childhood. His grandmother, Caroline Swayze had purchased the property he now owned in two parcels, one in 1856 and one in 1869. On a five acre plot she had purchased (I assume about 1860) directly opposite on the south side of the railroad, she had built a two-story frame house and large barn where, I have been told, she maintained a small estate. In the rear of the property was a family burying ground (long since evacuated) in a square of large white pines. The house was entirely destroyed by fire in 19??. 

When but a small child, I can recall frequent visits with Pop and Chester to the unoccupied old homestead and sleeping on the floor on blankets we had brought with us. We undressed by the flickering light of an old kerosene lamp casting weird shadows through the empty room and with the smell of stale air and age making our accommodations somewhat haunting to say the least. As the house was rented at times, the attic which contained many family heirlooms was kept locked. All were destroyed in the fire and I do regret that the folks had not had the foresight to preserve the old muskets and other antiques in a safer place. But what would be considered antiques today were simply ``older things'' to my folks at the time.

Pop loved the place and sorely missed the old house. Shortly after the fire, he built what we called a one-room shack (about the size of a large country chicken coop) in the orchard adjoining the site of the old house. Chester and I spent many happy weekends camping out there and playing with the children from a house which had been formerly the old Lakeland railroad station before the station stop was moved one mile east to its present location. Incidentally, Grandpa Morris was the agent of the Lakeland station for a couple of years prior to his death in 1991 and Grandma Morris continued on in that capacity for a short time after. But now I am losing the continuity of my story. I had planned to deal with events before my time in a supplement to this. 

The little old shack which housed us on those happy weekends was moved across the tracks in the oak grove and became one of three Pop built for campers and to make a little extra pocket money. I think his little shacks were among the forerunners of our modern motels. After Mabel and I had resumed ``normal lives'' again, we visited Ronkonkoma frequently. Many such visits will be mentioned as we get along with our story.

As I mentioned earlier, Mother and Dad Norseen had sold their home on 10\textsuperscript{th} Street and were renting. They missed the pride of home ownership and suggested that perhaps we could pool our resources and build a house. In the spring of 1925 we planned such a home and bought a lot on East 23\textsuperscript{rd} Street in the Flatlands section of Brooklyn. Dad knew a Swedish builder who we called in for consultation. He secured an architect and a contract was drawn up for a two family house -- five rooms and bath on each floor with a spacious cellar and toilet. As I had had little time to accumulate savings since Mabel's illness, Dad and Mother agreed to take up a second mortgage for us. So we were on the way toward home ownership, and for Mabel and me a heavy burden of interest and amortization charges with little money left for other things. We got a great kick from watching the house grow during that summer and fall. We literally inspected every piece of material going tino its construction. It really was a well built house, not because of our careful watch but because we had an honest builder. Mabel did a good job, too, in shopping for furnishings and interior decorating. The house was completed in December 1925 and we celebrated our first Christmas in that new home of ours at 1627 East 23\textsuperscript{rd} Street, Brooklyn. 

To all young people comes a great feeling of pride with ownership of their first home. Such pride in my opinion is greater only in the marriage ceremony itself and in the birth of the first child. To Mabel and me, one of the two greater feelings of pride -- the birth of our boy -- was to be withheld for several years. We were proud of that house. It was a palace in our eyes. Our parents were just as proud and happy. But I think their happiness was a reflection of our own. All parents want to know that their children are happy and gain happiness themselves through that knowledge. We do not realize that fact until we are parents ourselves and feel, telegraphically it seems, the happiness or unhappiness of our children. 

In the new neighborhood we made new friends and continued also to keep in close touch with old friends. Sally (Lund) Upham, a friend of Mabel since childhood, lived a block away on 22\textsuperscript{nd} Street. She introduced us into a group of young couples living in the neighborhood and we were welcomed into the fold. The men of the group formed a ping pong club and played one night a week alternating in the basements of those who had tables. It was not long before I, too, had a table. To this day, we speak of these friends as the ``ping pong'' crowd.

Shortly after we took up residence on East 23\textsuperscript{rd} Street, Wad and Marthe Smith bought a small one-family house nearby on East 16\textsuperscript{th} Street. Wad was climbing rapidly up the ladder of business success in General Motors and his work took him abroad frequently. When he was at home, however, we saw him frequently and had many good times together. They now had two boys, Jack who was six or seven years of age and Edgar who was a small baby just a few months old. They also had a young colored maid. At that time this was a real indication of affluence. The maid, Ida Winters, remained with the Smiths until after Wad's death many years later and was a real ``mammy'' to their children. 

In our new neighborhood, many of the girls of Mabel's sorority, and their husbands of course, lived nearby. Likewise, two of my old Richmond Hill friends, Harry Maass and his wife Billie, were conveniently close. Thus, we were in a wide circle of good companionship and caught up in a social life which we thoroughly enjoyed. It was not a pretentious social life by any means because we were all young married couples who had to watch the pennies. Despite the pitifully small amounts we could afford for social activities, we were rich in friendships. It was my opinion then, and still is, that aside from one's family, the greatest assets are a person's friends. They and members of the family provide the little world around which everything else revolves, and in the bigger world we call the Earth there are millions and millions of such compact little worlds in which inhabitants reap rich dividends in the form of love and friendship.

Now with additional expenses of the new home and expanding social obligations, I simply had to continue to attempt to build sorely needed material assets. I applied myself with even greater intensity to my job. It seems that when most discouraged with my progress and the progress of the business, I worked harder. Perhaps I was more loyal to the business than to myself. Through mail courses for home study and consistent reading of \href{http://en.wikipedia.org/wiki/Printer's_Ink}{Printer's Ink}, the ``business bible'' of that period, I sought the academic learning of business that I had missed. Looking back, however, the opportunity of becoming familiar with all areas of a small business together with the opportunity to understand people through contact with people in all phases of work in the plant and with the varied personalities of my customers gave me a broad practical education that stood me in good stead in later years. At the time, however, it seemed I was moving backward instead of forward. 

With the death of one of our older salesmen, I took over a part of his territory -- Philadelphia, Trenton, Southern New Jersey, Bridgeport, New Haven and some nearby areas. At the same time we inaugurated truck deliveries to Philadelphia and New Haven. This gave me the opportunity to increase our business in those markets because the customers would have no freight charges but we would have the expense of the haulage. A market cultivated for years by a man who had the respect and confidence of the trade is not easy for a new man to work. This I found out after a couple of trips. 

Fortunately, I met a fellow salesman, Charlie Schwartz, who lived in Philadelpia and was a son-in-law of the owner of the Goldenberg Candy Company, manufacturers of \href{http://en.wikipedia.org/wiki/Goldenberg's_Peanut_Chews}{Goldenberg's Peanut Chews}. Charlie was very popular in the trade and he took me ``under his wing'' and opened many doors of customers that would have been difficult for me to do at so early a time in my venture in that market. We became very friendly. I rode with him in his car when I worked the Philadelphia and Southern Jersey market, and we with me when he worked the markets in my home territory. Often times I would stay overnight at his home and would reciprocate by having him at my home. But in those first years in the Philadelphia market, I put  up in the old \href{http://digitalcollections.nypl.org/items/510d47da-550b-a3d9-e040-e00a18064a99}{Green's Hotel} which has long since been torn down. In our Southern Jersey trips, Charlie and I would stay at a hotel in Bridgetown, New Jersey. 

The New Haven-Bridgeport territory was a difficult one to work largely because I did it in one Saturday a month. With road conditions as they were in the mid-twenties it would take me three hours to reach New Haven -- and that with fast driving on those stretches of the old Boston Post Road where conditions permitted. As my first appointment was at 9 am, this meant that I would have to leave home at 6 am and after working New Haven, Bridgeport and Norwalk, arrive home at nine or ten o'clock at night. There could be no social activities for us on those Saturdays. These trips were particularly difficult during the winter months with snow and ice slowing us down. As can be seen from this illustration of my work, long hours were the rule rather than the exception, and to heighten my discouragement, DeBevoise's business was slowly declining. As conditions worsened with no hope of increasing my earnings, I took on Peanut Chews as a side-line and managed to pick up a few extra dollars in commissions.

Mr. DeBevoise had held out the hope over the years that he would eventually give the business to his key men. This expectation was the lure that kept us working at an abnormal pace. In 1927, after returning from my Philadelphia journey, I was told that he had held a meeting with key men and offered to \textit{sell} the business -- not including the buildings -- for \$100,000. I was also told that I was inculded in the group and would be appointed sales and advertising manager. I thought that now we could inject new ideas into the organization and really make some money. I was keen to take that gamble even though it meant an initial endorsement of \$5,000. [Illegible] My problem was to raise my share of the down payment. I had no cash. I approached a boyhood friend who I knew was making considerable money in a business he had taken over from his father and developed into a nationally known organization. To my surprise, he pleaded lack of funds. Finally, I reluctantly went to Pop and Wad Smith. Wad loaned me \$3,000 without a question, but Pop could only afford \$1,000. Despite my limited income, the notes were paid off at six percent interest within three years. It was a real struggle to meet those obligations and it could not have been done without the cooperation and sacrifices of Mabel. She made her own clothes and kept our household expenses at a bare minimum. At one time she sold magazine subscriptions door-to-door.

Our hopes for injecting new ideas into the company did not materialize. My associates were older men, steeped in the traditions of the past and unable to recognize the changes that were taking place in our industry -- changes that demanded new concepts of products and markets. As a result of this failure on the part of my partners to take action to meet new needs, the business went from bad to worse. By mid-1928 our sales were down to a half million dollars and I was selling one half of that total. The situation was bad and I was driving myself at a pace much too fast to be continued. 

The situation in which I found myself was ironic. Here we were near the tail end of the ``roaring twenties'' which will go down in history as a period of post-war inflation, rising costs and wages and wild speculation. With the exception of a brief recession in 1927, the country was prosperous and business in general was booming. These conditions were the source of our problem in the penny candy business. The value of money was declining as it always does in inflation and the penny was insignificant and in reality replaced by the nickel.

Grocery, cigar and 5 and 10$\cent$ chain stores were expanding in numbers and absorbing the business of the small stores. Five cent bar candies as well as package and bulk candies were on the counters of these stores and at reduced prices. The little candy stores (and the little grocery stores, too) were fast disappearing -- and with them our major market. 

With the thought that it might not yet be too late for a final attempt to make my associates understand the need for a major readjustment of our business, I started work on a detailed survey and analysis of our business which would support recommendations for needed action. This was an exhaustive study of product and market trends over the years. It took many hours of my ``spare'' time to get together the facts on production and sales of the more than one hundred manufactured items to make the accurate and convincing analysis of the need for action to save the business and move ahead to new goals. 

But, again, my associates were reluctant to face realities. They could not understand the facts of business life and still hoped for a miracle restoring old trends of consumer demand. The only solution for me was to get out of the business and seek a connection with a more progressive organization. I had been in business for 16 years (including two years in war service) and never had to look for a job. I became an avid reader of the classified section of the New York Times and spent my Sunday afternoons answering ads for sales executives in all kinds of business and mailing personalized form letters to a long list of companies with whom I would like to connect. However, few were interested in a fellow whose products were sold for pennies. 

One that held the most promise was an interview with the [Illegible], at the time a sales leader in the high-grade package trade. My appointment for an interview was set for a Friday afternoon in a suite at a New York hotel. Unfortunately, that was a day I was returning from Philadelphia where I missed my train and arrived panting and perspiring an hour late for the interview. I was received courteously but coldly by the President of the Marketing Manager. It was quite evident right at the start that I would not receive favorable consideration because of my tardiness. After the interview I received the usual polite note telling me that the job had been filled. 

Another interview was with an executive of General Foods with [Illegible] Francis who later became Chairman of that Corporation. I liked Mr.~Francis and would have enjoyed working for him but it was the policy of General Foods to have all their sales personnel start at the bottom as truck route salesmen. As the pay was about half of my pay at DeBevoise, I could not accept the job and still meet my financial obligations and maintain our existing living standards.

Just about the time my discouragement was at the lowest possible depth, I had the idea that if my survey and analysis of the DeBevoise business was as good as I thought it to be, there was no reason why it could not be used as a selling tool in my job search. I typed copies of the study and mailed it with t a covering letter to the \href{http://en.wikipedia.org/wiki/Young_\%26_Rubicam}{Young and Rubican}, a young and progressive advertising agency, and to \href{http://en.wikipedia.org/wiki/General_Motors}{General Motors} Export Company. General Motors was also a young and aggressive organization and from the stories of my two friends, Wad Smith and Marle Hale, was destined to be an outstanding industrial organization. Wad was Assistant to the President of the Export Company and Marle was Personnel Manager. Here was a situation that appeared to be made to order for me. But the big drawback was my friends. Because I was in a business entirely unrelated to automobiles and with a yearly volume equal to the daily volume of the Export Company, they felt I could not measure up to the standards of their organization. They were afraid to recommend me. They had their own reputations at stake and any failure on my part would be a reflection on their judgement. 

Both had discussed their thoughts with me and I knew just where I stood. As a matter of fact, I had given up all hope that I would get into GM. However, I did not realize the effectiveness of that Survey and Analysis. One of them, never found out which one, showed my letter and attachment to the President, \href{http://en.wikipedia.org/wiki/James_D._Mooney}{James D.~Mooney}, and he was interested. Now, my friends had the go-ahead. Late one night in mid-August, I received a telephone call from Merle telling me that I had a job and after the preliminaries of filling out application forms I was to start work on September 16, on Mr.~Mooney's staff at a salary of \$400 a month, which was fairly substantial in those days. the job was ``special studies.'' I could be thankful indeed that I had toiled so hard to prepare that survey. And this thankfulness was doubly felt with the receipt of a letter from Young and Rubican requesting that I drop in for an interview. They were seeking a man to go into the plants of those of their clients where there seemed to be deficiencies in products or sales and make recommendations for improvement. In other words to do what my survey was planned to do for DeBevoise. 

With the GM job assured, I was faced with the problem of breaking the news to my associates at DeBevoise and making arrangements for the return of my investment. My departure would be a bad break for them, not only because of the sales I was bringing in but also because of the fact that my contract specified of the repurchase of my stock. The Company could ill afford this additional financial loss. As there was no cash, I agreed to take a series of notes calling for monthly payments to me. Despite their unwillingness to appreciate the true situation regarding their business practices, I felt a real pang in leaving them. Four years later the business failed, thus confirming my appraisal. I was the only member of the rim to recover my investment.

\chapter{Beginning at General Motors}

After a short vacation, I reported for a new job at GM on September 16, 1929. It is a coincidence that I started my job with DeBevoise on September 16, 1912 and as I note these facts the date of this writing is June 16, 1963.And this sixteenth of June is an historic date because it is the 73\textsuperscript{rd} anniversary of the marriage of my parents. An event which sparked the beginning of our generation of the Morris family -- Chester in 1891, James (Alfred) in 1893, Vada in 1897 and Kenneth in 1908. This anniversary date stirs memories of many happy reunions but let's leave that in the interest of continuing my business story. I will deal with those occasions in a later section. 

On that morning of September 16, 1929, I entered the General Motors building with considerable trepidation. How could I, a man from little business, fit into this big business organization? How could I compete with the men who had helped to make it grow to such a size? But I had asked for a job and here I was on the threshold of what could be a new and exciting career. So buck-up, old man, keep your eye on the ball and show them you can compete. Such thoughts were interrupted by a warm greeting from a personnel man, Bill Winslow, who ushered me into a big window corner office on the 17 \textsuperscript{th} floor containing two desks -- one with a name plate block reading ``James A. Morris.'' I was thrilled to say the least. Compared with my cramped quarters in front of the dirty sink and behind that old stand-up bookkeepers desk in the DeBevoise office this was just too good to be true. But there was a name on the desk. Maybe, those last few years in the candy business were a nightmare and I had awakened in the office where I belonged. 


























\end{document}