\documentclass[12pt]{book}              % Book class in 11 points
\parindent0pt  \parskip10pt             % make block paragraphs
\usepackage{hyperref}
\usepackage{wasysym} % Needed for cents symbol
\usepackage{float}
%\usepackage{footnote}
%\makesavenoteenv{tabular}
%\makesavenoteenv{table}

%\raggedright                            % do not right justify

\title{\textsc{Family Portrait}\\ The Memoirs of James A. Morris}    % Supply information
\author{J. A. Morris}              %   for the title page.
\date{New York, 1962}                           %   Use current date. 

% Note that book class by default is formatted to be printed back-to-back.
\begin{document}                        % End of preamble, start of text.
\frontmatter                            % only in book class (roman page #s)

Transcribed by Brett Morris, great-grandson of the author, in summer 2014. This is an lightly edited reproduction of the original typed manuscripts with hand-written annotations and corrections. 

In some places, the hand-writing is too poorly copied to be read. When large bits of text are missing that cause breaks in the narrative, they will be marked with the word: [Illegible]

\maketitle                              % Print title page.
\tableofcontents                        % Print table of contents
\mainmatter                             % only in book class (arabic page #s)
%\part{A Part Heading}                   % Print a "part" heading
\chapter*{Forward}                % Print a "chapter" heading
%Most of this example applies to \texttt{article} and \texttt{book} classes
%as well as to \texttt{report} class. In \texttt{article} class, however,
%the default position for the title information is at the top of
%the first text page rather than on a separate page. Also, it is
%not usual to request a table of contents with \texttt{article} class.
% 
%\section{A Subheading}                  % Print a "section" heading
%The following sectioning commands are available:
%\begin{quote}                           % The following text will be
% part \\                                %    set off and indented.
% chapter \\                             % \\ forces a new line
% section \\ 
% subsection \\ 
% subsubsection \\ 
% paragraph \\ 
% subparagraph 
%\end{quote}                             % End of indented text
%But note that---unlike the \texttt{book} and \texttt{report} classes---the
%\texttt{article} class does not have a ``chapter" command.

I have always had the urge to write something of longtime interest to others. During a long business career I did a lot of writing -- studies, reports, suggestions -- of the current interest only. They were communications that seemed very important at the moment. But I daresay very few could be found at this writing (January, 1962). If available, they would have little meaning now. Perhaps a suggestion on business policy is still timely and hauled out of the back of an office file for reference. Of, maybe, a letter to family or friends contained a helpful comment that warranted storing in that packet of memories in the attic trunk. However, there are few communications in the average person's life that stand the test of time.

This was emphasized very strongly in my attempt to reconstruct the family genealogy. The record goes back for more than 300 years. But it is a record of births and deaths with a bare sprinkling of facts on which to evaluate the character, life and works of the individuals. Perhaps the urge to set down some words that would be of enduring value as a family portrait sketch for the younger offshoots is due to the fact that I, like others of my generation, will also be a mere statistic before many years. This urge to preserve a background of the family is much stronger than my lifetime reluctance to look back. 

[Illegible]

So, here is the beginning of a story that might, I hope, give life to some of the statistics on the recent generations of my family. Because it must necessarily revolve around the one I know most about, it is written in the first person. 

\chapter{Early Childhood: ``The Story Begins''}

For me, the story begins with a statistic -- April 6, 1893 -- in a two-story and basement frame house on Buffalo Avenue, Brooklyn. Around the corner at 844 Herkimer Street in the backyard garden of a more pretentious brick house, my father's young cousin, Caroline Halliday, was playing and keeping a watchful eye on the upper rear windows of the Buffalo Avenue House. She was watching for a signal that her mother, Grandma Morris' sister, was needed. The signal, a sheet hung out a window, came and Aunt Addie rushed to the Buffalo Avenue house and helped with my birth. Eighteen months earlier my parents had welcomed their first child -- Chester DeVere -- my big brother to the world. Four years later in 1897 my sister, Vada was born in the Buffalo Avenue house.

My father, Isaac J. Morris, was 25 years of age when I was born and my mother, Birdella LaBonte Morris, 23. Whether they wanted me so soon after the first child is a moot question. But I came along anyway and Mom told me in later years that I was a cry baby, a thumb sucker and an apron-string hanger-on until I was old enough to go out and rough-it-up with my playmates. I do know that I was strongly attached to her. 

I can still recall the misery experienced when she left me for the first time to visit her parents in Albany. We lived in Morris Park then and the memory of my walk with her to the railroad station is vivid. I was crying and she would pause to dry my eyes and console me. To this day, the sound of a train whistle in the night brings back the acute feeling of loneliness I felt that first night she was away. 

I have no recollection of my babyhood in the Buffalo Avenue house. All of my memories of the neighborhood stem from childhood visits to Herkimer Street with my parents after we had moved to Morris Park. Aunt Addie was a widow with three children -- Isaac, Carrie and Walter. Grandma Morris (Ann), also a widow, lived next door with her brother, David Swayze. At that time the neighborhood was upper middle class. To us it seemed like millionaire's row for we were quite poor. The street was cobble-stoned and I can still hear the click-click of horse shoes and the lurching screech of iron-rimmed wagon wheels on the stones. 

The gardens in the rear yards stretched back for maybe 150 feet and we enjoyed playing in them. There were a couple of peach trees with low-hung branches that tempted us to snatch at the big, luscious fruit when Aunt Addie could not see us. 

The Haliday house was a lively one. The two boys were always playing pranks on their mother and sister, and on Chester and me when we visited them. Cousin Ike was the older and more serious of the two and highly opinionated -- a fact that made him rather forbidding in later life. Walter and Carrie were really fun-loving and laughter was the order of the day -- even more so when Arther Blinn, Pop's nephew, visited the Halidays. I think Arthur's mother, Caroline Morris Blinn, died during his birth. It was a home we loved to visit. But it was not without its tragedy. 

Aunt Addie had the problem of raising three children. To help bolster a limited income, she took in boarders. One of them, a handsome real estate broker, courted and married her in 1901. A short time later he disappeared and was not heard of again. Although she never said so, we believe she expected he would return and for that reason has remained in the Herkimer Street house. At this writing, she is well into her eighties and still there in a much run-down neighborhood -- a sad contrast to its stately refinement of her childhood. 

Despite little formal education, Cousin Ike was successful in business. He was associated with an estate and at one time as part of his responsibilities managed the \href{http://en.wikipedia.org/wiki/Hotel_Theresa}{Hotel Theresa} in Harlem. That section of Harlem, now entirely negro, was a very fine neighborhood. It was really a big treat to dine with Cousin Ike and his wife Birdie. Mother would polish us up, put on her very best clothes and admonish us on our table manners. Cousin Ike later became night manager of the New York Stock Exchange Clearing House, a job he kept until his retirement. 

Grandma Morris, Uncle Dave and Aunt Addie have long since passed away. More recently, Cousins Arthur, Ike and Walter have died. Walter's son, Walter J., is living in Rockville Centre. 

As mentioned earlier, I have no recollection of my first home in Brooklyn. We moved to Morris Park (\href{http://en.wikipedia.org/wiki/Richmond_Hill,_Queens}{Richmond Hill}, south of the railroad) when I was about four years old. Pop was a machinist in the Long Island Railroad shops in that village. The new home was a far-from-pretentious flat in a three-story six-family house on the northwest corner of Briggs Ave (now 117th Street) and Chichester Avenue (now 95th Avenue). There was only one house directly opposite on the avenue. In the rear of our flat was a fenced-in yard and beyond that open fields almost to Atlantic Avenue. This is where we played ball when we were older. 

Morris Park was really a country village in those days. Five blocks south, Liberty Avenue was ``the end of the world.'' From there we walked through the woods (now Glen Morris) to the old water holes in the swamps of the Aqueduct. And we did not need bathing trunks for our swimming, the high grasses and cat-o-nine tails\footnote{a.k.a \href{http://en.wikipedia.org/wiki/Typha_latifolia}{\it Typha latifolia}, a wetland weed} hid us from view even if there was any one within hailing distance. 

Two events are the earliest recollection of my childhood. Both occured at the first home in Morris Park. I was just a little toddler playing in the yard while Mom hung the wash. An Irish woman leaning out an upper window shouted that my pants were falling open (actual words censored). If a baby can be embarrassed, I was and so was Mom. It was peculiar that an incident like that can be remembered while other more important ones cannot. The other event was at a Thanksgiving or Christmas party in our flat. The Halidays and Grandma Morris were there and my little baby sister Vada was the center of attention. In some way, one of the glass dessert dishes was cracked and I swallowed a small chip of glass. Pandemonium broke loose with much shouting and wringing of hands probably brought about by my wailing. Finally some one thought of calling a doctor. He filled me full of crackers or bread, topped off with a large dose of castor oil. Evidently they found the glass. 

\section*{Pop}

Pop was an apprentice machinist in the railroad shops at Albany or Rensselaer when he courted Mom. [Illegible] I do know that he worked hard and had long hours. As I recall, the deep-throated, long-carrying shop whistle awakened the entire village at seven each morning, blew at 12 noon for a half hour lunch period and at 6 p.m. for closing. That was a 10.5 hour day for the shop personnel. His pay check for a six day week was \$12 or \$15. Compared with present day hours and pay that certainly was unbelievable. 

It was a monotonous treadmill of work with little time left for anything else and with no future. Pop was handicapped by lack of formal education. He had not completed grammar school but was an excellent mechanic and of an inventive turn of mind. He sought to lift himself through inventions and by going into business for himself. He was not successful in either direction. 

At one time he felt there was a market for a convenient hand washing compound. From his daily experience he knew the difficulty of washing grease and grime from his hands at the end of a day's work. His idea was a compressed washing powder about the diameter of a quarter and a quarter inch thick, packed in rolls like the old candy \href{http://en.wikipedia.org/wiki/Necco_Wafers}{Necco Wavers}. Held in the palms of the hands under the water tap it would dissolve and with a little rubbing remove the grease. He advertised for shop workers with little success. [Illegible] Later the big soap companies introduced hand washing compounds in pase form which met with the success that Pop had so much sought. 

Another of his many inventions was a collapsible wooden crate for packing onions and other vegetables -- that could be returned to the farmer or produce dealer and reused. He patented the idea, incorporated a company under the name {\it American Crate Company} and sold stock to his friends. Then the promoter sold Pop on teh idea that he could sell the crate to the onion growers of Texas and departed with the remaining cash. Years later he sent shares of stock in an oil exploration company to Pop and to the stockholders of the Crate Company. They turned out to be as worthless as the crate company stock and ended another dream for Pop. 

We can return to our early days in Morris Park with Pop's venture in the retail grocery business. Three doors down the block from our flat was a little frame house, the front room of which had been converted into a store. Pop left the railroad, moved the family into the little house and opened a grocery store advertised as ``The Little Store with the Little Prices.'' One thing I recall about the house was the grape arbor forming a leafy corridor to a two-holer outhouse. Perhaps the corridor was not too long, but one time I didn't quite make its entire length much to the consternation of Mom and my own embarrassment. By way of extenuation it may be said that I was only four or five years old. 

A very short time later, Pop arranged to have a new store built on the lot immediately adjoining the little store. The structure was built on the entire width of the lot. It had a built-in driveway to a stable in the rear of the house. The store and rear storeroom took up the remainder of the first level. From the storeroom a staircase led to a very nice apartment above. Pop continued in business here for a few years until his leniency on credit brought the business toppling down.

An old oder book (I don't know if it was for delivery from the little store or from the new one) discloses some rather startling prices when compared with those today. Here are a few examples: 

\begin{table}[H]
\centering
\begin{tabular}{l r}
3.5 pounds of sugar & 17\cent \\
1 pound of coffee & 36\cent \\
1 loaf bread & 5\cent \\
1 can condensed milk & 10\cent \\
1 quart milk & 5\cent \\
1 gallon Kerosene oil & 13\cent \\
1 lamp wick & 2\cent \\
1 quart white onions & 12\cent \\
1 peck\footnotemark potatoes & 18\cent \\
1 bushel\footnotemark coal & 17\cent \\
\end{tabular}
\end{table}
\footnotetext{15 pounds}
\footnotetext{80 pounds}

\chapter{Memory begins to crystallize}

It is from this place that my memory begins to crystallize. Names of playmates, the rough and tumble fights of childhood, the short cut across the fields to the primary school on Elm Street, and the many other things that made their impressions on a developing mind. I would like to mention a few incidents not only for the nostalgic interest they hold for me but as a sort of factual background for appraisal of the effect on our character. 

First, let me say that my sister Vada was born in this house in 1897. I don't remember very much about her entrance, only the few times Mom asked me to push the baby carriage. I do remember the admiration she drew from the relatives when they visited Morris Park. She was a beautiful baby and grew into a beautiful girl and woman.

Because Pop's name was Issac, two of the neighborhood boys delighted in calling me ``Ikey.'' One of the boys, George Washington, was colored and the other, Tommy Givins, Irish. They were a little older and I was dreadfully afraid of them for a long time. Finally the taunts of Tommy hurt so much that I forgot I was afraid and we fought it out in the fields coming from school. At the end we were both sobbing and I may have won by a very small margin. Later I managed to fight it out with George also. With those two childhood fights came confidence in my physical ability to give and take -- a confidence that saved me from many other boyhood fights. I had learned that a bully thrives on the fears of his victim. Take away the power to frighten and the bully is deflated. 

I learned, however, through a different kind of incident that there are other kinds of fears. For some reason or other my teacher in old 53, I think it was the first grade, locked me in a closet. I was frightened to the point of hysterics. When she opened the door, I was on the floor with my nose to the thin strip of light at the bottom of the door. For many years after, I was afraid of the dark and to this day fear closed places. 

Our new home was lighted with gas continuously by ``pay-as-you-go'' meters. One Sunday evening the lights went out and Pop asked me to go to the cellar and drop a quarter in the meter. To do so I had to go downstairs, through the dark store room and on down to the even darker cellar. I was searching for the slot in the meter when something scrambled away. I ran upstairs screaming and Pop went down and discovered a cat in the cellar. That was another experience that made me afraid of the dark.

Mom was deeply religious and her training was reflected in our early formation of character. All of her social contracts were made at the First Methodist Episcopal Church of Morris Park and her children were drawn into the circle. On Sundays we attended the morning services with her. We could be found at Sunday school in the afternoon and after supper at the evening service with Mom. I can still recall sitting beside Mom and singing such hymns as {\it Jesus Lover of My Soul}, {\it Lead Kindnly Light} and {\it Rock of Ages}. When we were a little older we joined the Epworth League and attended those meetings just before the evening service. Sunday was indeed a crowded day. 

Wednesday was also a special day for Mom. On that evening she went to prayer meeting with a clsoe friend and neighbor, Mrs. Bradford Wicks. Pop rarely attended any of teh church services. 

Mom did not believe in any frivolity on Sundays. She would not let us play ``catch'' or any games and it would have taken a real emergency to allow us to ride a trolley car on that day. Sunday was a day we wore our blue serge suits and didn't dare get them soiled. When they became too shiny for Sunday and holiday we used them for school. 

The Methodists were really strict in those days. Each service was an evangelistic meeting in itself with much talk of hell and brimstone. Dancing was frowned upon and drink was a curse. Mom would get real excited when a Catholic neighbor would pass the house with a tin pail on the way to a nearby saloon for a pint of beer. This was called ``rushing the growler'' and the tin pails were large enough to hold far more than a pint. She could not understand either why the Catholics would permit ball playing on Sundays or hold picnics or other festive affairs. 

With no movies, radio, television or automobiles, the church -- no matter what the denomination -- was the social center of the community. Apart from religious services, in the summer there were strawberry festivals on the lawns of the church or private homes, trolley rides to Coney Island and picnics. In the winter there were straw rides in horse drawn sleighs or wagons, or house parties of the various societies.For the children there were two big events. One was the Anniversary Day Parade in early June with ice cream and cake awaiting them in the church basement at the end of the march. With bands playing such tunes as {\it Onward Christian Soldiers} and with the children decked out in their best clothes carrying banners and American flags, it really was a gala event. 






\end{document}